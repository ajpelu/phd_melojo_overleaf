% !TEX root = ../my-thesis.tex
%
\chapter{\textcolor{ctcolormain}{Conclusiones}}\label{sec:conclussions}
\newpage

\section{Conclusiones}\label{sec:conclussions:spa}

\begin{enumerate}
\renewcommand{\labelenumi}{\textbf{\textcolor{ctcolormain}{\arabic{enumi}.}}}
    \item El aumento de temperatura registrado en las últimas decadas está ocacionando cambios en la dinámica de los ecosistemas de montaña. Mediante el uso de un sistema de ontologías y datos procedentes de teledetección, se ha explorado la relación entre los cambios observados en la dinámica de la nieve y los cambios observados en la productividad de los robledales de Sierra Nevada. Hemos observado la concurrencia de un aumento significativo en la productividad primaria de verano para los robledales, y un adelanto significativo en la fecha de fusión de la cubierta de nieve. Esta modificación en los patrones de disponibilidad de agua debido al cambio climático parece estar afectando a la productividad estacional de los robledales. Los robledales donde la nieve presenta una tendencia significativa a retirarse antes, presentan un incremento en la productividad primaria. Este acoplamiento entre las tendencias de producción primaria y las de duración de la nieve es más patente para las poblaciones de robledales occidentales de Sierra Nevada, donde el 60\% de los pixeles muestran un adelanto en la fecha de retirada de la nieve y un aumento de la productividad en verano. 
    
    \item Las poblaciones de \Qpy del límite meridional de su distribución situadas en zonas de montaña, no son ecológicamente homogéneas, ni por sus condiciones ambientales ni por su composición de especies vegetales. Hemos observado que la disponibilidad de agua es una variable clave para explicar la distribución de \Qp y la diversidad florística de sus comunidades acompañantes en Sierra Nevada. Basándonos en variables ambientales, hemos identificado tres grupos de poblaciones de \Qp en Sierra Nevada. Estos grupos además presentan diferencias respecto a la diversidad vegetal. Se encontró una notable coincidencia entre la agrupación de poblaciones derivada del análisis de las variables ambientales y la ordenación de las poblaciones según la composición de especies. La diversidad de comportamientos ecológicos de las poblaciones de \Qp en este borde posterior de distribución, son consistentes con la alta diversidad genética que muestran las poblaciones de este roble en Sierra Nevada. La identificación de las diferencias entre las poblaciones de roble dentro del borde posterior con respecto a las variables ambientales puede ayudar a planificar la gestión forestal y las acciones de restauración, particularmente considerando la importancia de algunos factores ambientales en aspectos ecológicos clave.
    
    \item A pesar de las fuertes limitaciones de reclutamiento descritas para \Qp, hemos constatado que está ocurriendo un proceso de recolonización natural de los cultivos por parte de \Qp en Sierra Nevada. Este proceso de recolonización está siendo diferente entre las poblaciones de robledal. Ni la estructura del bosque circundante ni la abundancia de \emph{Garrulus glandarius}, el principal dispersor de bellotas,  varió significativamente entre las poblaciones de robledal. Las diferencias en los patrones de recolonización parecen estar relacionadas con diferentes historias de gestión, anterior y posterior al abandono del cultivo. Una mayor presión herbívora tras el abandono de las tierras de cultivo parece limitar la expansión del bosque hacia las tierras de cultivo abandonadas. 
    
    \item Hemos constatado que el robledal está capaz de colonizar dentro de la banda altitudinal donde habita, por lo que sería interesante realizar estudios que analicen si el robledal es capaz de colonizar por encima de su límite altitudinal. 
    
    \item Los robledales de Sierra Nevada, al igual que los del resto de la Península Ibérica, han experimentado un aumento de la biomasa total en las últimas décadas. El abandono de las actividades tradicionales y el éxodo rural son los principales factores que explican la densificación y la expansión de los bosques, especialmente en regiones montañosas como Sierra Nevada. 
    
    \item 


\end{enumerate}


Total estimated biomass values in Sierra Nevada oak woodlands ranged from 147.27 -- 157.13 \mgha

Los datos estimados de stock de carbono para los robledales de Sierra Nevada son superiores a los observados para otras zonas. Estas diferencias parecen estar relacionadas con ... explicar por: los bosques de robledal de SN son relativamente jóvenes (debido al intenso uso antrópico al que han estado sometidos) en comparación con otros robledales  y con la disponibilidad de agua 



Por lo tanto, podemos considerar que muchos de los bosques de robles de Sierra Nevada son relativamente jóvenes, lo que podría explicar el alto potencial de acumulación de C obtenido en nuestro estudio, ya que se ha demostrado que los bosques creados como resultado de cambios drásticos en el uso de la tierra exhiben tasas de crecimiento más rápidas, y por lo tanto una mayor acumulación potencial de C, que los bosques preexistentes \tocite{VilaCabreraetal2017NuevosBosques}. Las diferencias en las reservas de carbono entre los jóvenes montes de roble y los bosques maduros se registraron para otras especies \emph{Quercus}

Disponibilidad de agua 

Therefore water availability does not seem to be strongly limiting the oak-growth in this mountain region. In fact, in the last decades positive trends have been observed for greenness and secondary growth of oak woodlands in Sierra Nevada \autocite{GeaIzquierdoCanellas2014LocalClimate,PerezLuqueetal2020LanduseLegacies,RubioCuadradoetal2018AbioticFactors} suggesting that this mountain range could act as an ecological refugee for this species. Thus those positive growth trends could explain the high values of carbon sequestration obtained in our study.

Existen diferencias en cuanto al potencial de secuestro de Carbono entre las poblaciones de robledal de Sierra Nevada. Aquellas poblaciones que han estado sometidas a menos perturbaciones antropogénicas, presentan una estructura de bosque estable con altos valores de biomasa, y por tanto, con mayor potencial de secuestro de carbono. 






Las menores perturbaciones antropogénicas han dado lugar a bosques bien conservados, con una mayor diversidad de especies \Nautocite{PerezLuqueetal2021DiversidadEcológica}, y también a una estructura de rodales estable con valores altos de biomasa

We found that oak woodlands of the Northern cluster showed stands with taller and greater trees (\tabref{tab:carbon:compara}), and also high values of biomass (Figures \ref{fig:carbon:mapas-wt} and \ref{fig:carbon:schema}). It could be related with lower intensity of anthropogenic disturbances in comparison with the other oak woodlands, mainly because the northern oak woodlands had greater protection during the second half of the last century \autocite{JimenezOlivencia1991PaisajesSierra}, and currently also have the highest level of legal protection within the protected area \autocite{Anonymous2011Decreto238}. The less anthropogenic disturbances have resulted in well-conserved forests with a greater species diversity \autocite{PerezLuqueetal2021EcologicalDiversity}, and also a stable stand structure with high values of biomass (Figures \ref{fig:carbon:mapas-wt} and \ref{fig:carbon:schema}). For other species of \emph{Quercus} it has been observed that forests with less disturbance have a higher potential for carbon storage \autocite{BalboaMuriasetal2006CarbonNutrient,Cotillasetal2016AbovegroundBelowground,Stojanovicetal2017ForecastingTree}.















\paragraph{\emph{Cuantificar el papel de los robledales de Sierra Nevada como sumidero de carbono y analizar su tendencia temporal}} \mbox{} \\
El secuestro de carbono es uno de los servicios ecosistémicos más relevantes que proporcionan los bosques mediterráneos, siendo un indicador de la capacidad del ecosistema para contribuir a la regulación del clima. Los robledales, como ecosistemas mediterráneos representan un sumidero de carbono. El objetivo de este capítulo es cuantificar la capacidad  de secuestro de carbono de los robledales de Sierra Nevada, situados en el borde de su distribución, explorando las posibles diferencias entre las poblaciones de robledal dentro de esta región montañosa. Asimismo estamos interesados en analizar la evolución temporal de este servicio ecosistémico 

\paragraph{\emph{Analizar los efectos del cambio climático en la productividad de los robledales en Sierra Nevada}} \mbox{} \\
Pretendemos evaluar como las alteraciones en los patrones de disponibilidad hídrica debido al cambio climático pueden afectar a la productividad de esta formación forestal. Sabemos que los robledales presentan una estación de crecimiento bien definida y centrada en la estación estival. Por ello es de interés evaluar la productividad de los robledales (utilizando índices de vegetación obtenidos a partir de imágenes de satélite) frente a cambios en los patrones de disponibilidad hídrica. Nos centraremos en analizar modificaciones en la cantidad de agua (disminución de la disponibilidad de agua debido a eventos de sequía) así como alteraciones en la distribución temporal de la disponibilidad de agua (\emph{p.ej.}: adelantos en la fusión de la cubierta de nieve).

\paragraph{\emph{Evaluar la resiliencia de los robledales en su borde de distribución frente a eventos de sequía}}\mbox{} \\
El cambio global supone un reto para los ecosistemas forestales localizados en el borde de su distribución debido a su vulnerabilidad a los eventos de sequía. Nuestro objetivo es analizar la resiliencia a varios eventos de sequía de las poblaciones relíctas de \Qp situadas en Sierra Nevada. Para ello, combinaremos información de teledetección y métodos dendroecológicos para evaluar el impacto de la sequía tanto en el verdor de la vegetación (como indicador del crecimiento primario) como en el crecimiento radial de los árboles (como indicador del crecimiento secundario).

\paragraph{Identificar y cuantificar los servicios ecosistémicos proporcionados por los robledales en Sierra Nevada}\mbox{} \\
Realizaremos una revisión de los principales servicios ecosistémicos proporcionados por los robledales combinando revisiones bibliográficas con conocimiento experto y datos de Sierra Nevada, con el objetivo de poner de relieve los diferentes servicios ecosistémicos que proporcionan estas formaciones para incorporarlas a las estrategias de gestiòn de estos ecosistemas. 



\section{Conclusiones}\label{sec:conclussions:en}

\begin{enumerate}
\renewcommand{\labelenumi}{\textbf{\textcolor{ctcolormain}{\arabic{enumi}.}}}

    \item A natural recolonization of abandoned croplands by \Qp is occurring in the rear edge of the distribution of this oak species. Oak juvenile abundance in the abandoned croplands varied between study sites. Neither surrounding-forest structure nor the abundance of jays varied significantly between study sites. The differences in the recolonization patterns seem to be related to differences in the previous- and post abandonment management. A higher herbivory pressure after cropland abandonment seems to limit the forest expansion towards abandoned croplands.
    
    
\end{enumerate}