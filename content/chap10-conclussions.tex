% !TEX root = ../my-thesis.tex
%
\section*{Conclusiones}\label{sec:conclussions:spa}

\begin{enumerate}
\renewcommand{\labelenumi}{\textbf{\textcolor{ctcolormain}{\arabic{enumi}.}}}

    \item Las poblaciones de \Qpy situadas en límite meridional de su distribución y que se localizan en zonas de montaña, no son ecológicamente homogéneas, ni por sus condiciones ambientales ni por su composición de especies vegetales. En Sierra Nevada, la diversidad de "comportamientos" ecológicos que muestran las poblaciones de \Qp, es consistente con la alta diversidad genética de dichas poblaciones. 
    
    \item La disponibilidad de agua es una variable clave que explica la distribución del roble melojo en Sierra Nevada, así como la diversidad florística de las comunidades vegetales que le acompañan. 
    
    \item Para Sierra Nevada, se han identificado tres grupos de poblaciones de \Qp. Estos grupos de poblaciones presentan diferencias respecto a la diversidad vegetal que albergan. Se ha encontrado una notable coincidencia entre la agrupación de poblaciones derivada del análisis de las variables ambientales y la ordenación de las poblaciones según la composición de especies. 
    
    \item La identificación de diferencias, tanto en condiciones ambientales como en composición florística, entre los grupos de poblaciones de robledal dentro del límite sur de su distribución, puede servir de ayuda en la planificación de la gestión forestal y las actuaciones de restauración de estas poblaciones, particularmente considerando la importancia de algunos factores ambientales (\emph{e.g.} disponibilidad de agua) en aspectos ecológicos clave.
    
    \item A pesar de las fuertes limitaciones de reclutamiento descritas para \Qp, se ha observado la existencia de un proceso de recolonización natural de los cultivos de montaña abandonados por parte de \Qp. El robledal está siendo capaz de colonizar zonas abandondas dentro de la misma banda altitudinal donde habita. Teniendo en cuenta las previsiones de cambio climático, sería interesante estudiar la capacidad de colonización por encima de su límite altitudinal. 
    
    \item El proceso de recolonización de cultivos abandonados por parte de \Qp está condicionado por el uso previo y posterior al abandono del cultivo. En Sierra Nevada, se han observado diferencias en la recolonización de los cultivos por parte del \Qp. Diferentes historias de manejo condicionan la abundancia de regeneración dentro de los cultivos abandonados. Así, una mayor presión herbívora tras el abandono de las tierras de cultivo parece limitar la expansión del bosque hacia éstas tierras.
    
    \item Los robledales de Sierra Nevada, al igual que los del resto de la Península Ibérica, han experimentado un aumento de la biomasa total en las últimas décadas. La biomasa total estimada para los robledales de \Qp en Sierra Nevada asciende a 9.94 Tg (1 Tg = 10$^{12}$ g), lo que representa un secuestro potencial de CO\textsubscript{2} de 17.33 Tg. 
    
    \item Los robledales de Sierra Nevada, a pesar de estar situados en el límite sur de su distribución, presentan un alto potencial de secuestro de Carbono. Los datos estimados de stock de Carbono para los robledales de Sierra Nevada son superiores a los observados para otras zonas. Estas diferencias parecen estar relacionadas con el hecho de que los robledales de Sierra Nevada son relativamente más jóvenes en comparación con otros robledales (debido al intenso uso antrópico al que han estado sometidos). No obstante sería necesario realizar estudios que analicen en profundidad las diferencias en la capacidad de secuestro de carbono de esta formación a lo largo de su rango de distribución, y los factores que las explican. 
    
    \item Existen diferencias respecto al potencial de secuestro de Carbono entre las poblaciones de robledal de Sierra Nevada. Aquellas poblaciones que han estado sometidas a menos perturbaciones antropogénicas presentan una una mayor riqueza estructural, que se ve reflejado en valores mas altos de biomasa, y por tanto, en un mayor potencial de secuestro de carbono. 

    \item El aumento de temperatura registrado en las últimas decadas está ocasionando cambios en la dinámica de los ecosistemas de montaña. Mediante el uso de un sistema de ontologías se ha observado la concurrencia de cambios en el patrón de innivación y en la productividad de los robledales de Sierra Nevada. En las zonas donde se ha detectado un adelanto en la fecha de fusión de la nieve, también se ha observado un aumento significativo en la productividad primaria de verano para los robledales. Esta modificación en los patrones de disponibilidad de agua debido al cambio climático parece estar afectando a la productividad estacional de los robledales. Este acoplamiento entre las tendencias de producción primaria y las de duración de la nieve es más patente para las poblaciones de robledales occidentales de Sierra Nevada, donde el 60\% de los pixeles muestran un adelanto en la fecha de retirada de la nieve y un aumento de la productividad en verano. 
    
    \item Los robledales proporcionan una gran cantidad de servicios ecosistémicos. Además del papel que presentan estos bosques como proveedores de servicios de regulación (\emph{e.g} sumidero de Carbono) o de provisión (\emph{e.g.} uso de su madera para el envejecimiento del vino), se ha puesto de manifiesto la existencia de un gran número de servicios ecosistémicos culturales proporcionados por los bosques de \Qp. La revisión de servicios ecosistémicos realizada para los robledales de Sierra Nevada, combinando criterio experto junto con datos locales, proporciona a los gestores del territorio una información muy valiosa que puede ayudar en la planificación de actuaciones de gestión y conservación de esta formación forestal. 
    
    \item Los robledales localizados en el límite sur de su área de distribución muestran una gran resiliencia a la sequía. El crecimiento primario y secundario de esta especie, aún siendo muy sensible a la disponibilidad de agua, mostró una alta resiliencia a corto y largo plazo a los eventos de sequía. Los altos valores de resiliencia observados sugieren que las poblaciones de robledal de Sierra Nevada se encuentran en el borde posterior geográfico pero no climático ni ecológico. 
    
    \item En los últimos años se ha observado una tendencia positiva en la productividad primaria y en el crecimiento secundario de los robledales de Sierra Nevada. Esto podría estar relacionado con un efecto positivo del calentamiento para las especies que se encuentran en lugares de alta elevación limitados por el frío. Asimismo, es importante destacar que esta respuesta es diferente a la tendencia negativa de crecimiento que cabría esperar para las poblaciones situadas en el borde sur de su distribución, tal y como se ha observado para otras especies templadas y mediterráneas. 
    
    \item La resiliencia a la sequía que muestran los robledales de Sierra Nevada no es espacialmente homogénea, debido a las diferencias en las condiciones ecológicas y a los legados del uso del suelo. La gran variabilidad en las respuestas entre sitios muy cercanos geográficamente, y que se manifiesta a través de diferencias en el crecimiento, distinta sensibilidad climática y diferente resiliencia a la sequía, parece indicar que las respuestas a la sequía son dependientes del sitio y pueden variar drásticamente en gradientes espaciales extremadamente estrechos, como los que ocurren en regiones de montaña. 
    
    \item El concepto de borde posterior de distribución de una especie, necesita considerar los usos del pasado como un aspecto más, además de los aspectos geográficos, climáticos y genéticos, sobre todo en regiones con una larga tradición de modificaciones antropogénicas del territorio.  
    
\end{enumerate}

\section*{Conclusions}\label{sec:conclussions:en}

\begin{enumerate}
\renewcommand{\labelenumi}{\textbf{\textcolor{ctcolormain}{\arabic{enumi}.}}}

    \item The rear-edge populations of \Qpy located in mountain areas are not ecologically homogeneous, neither for their environmental conditions nor for their plant species composition. The diversity of ecological behaviors for \Qp populations in Sierra Nevada are consistent with the high genetic diversity shown by populations of this oak in this rear edge.
    
    \item Water availability is a key variable explaining the distribution of melojo oak and the floristic diversity of their accompanying communities in Sierra Nevada.
    
    \item For Sierra Nevada, three clusters of \Qp populations were identified based on environmental variables. These population groups also differ in terms of the plant diversity they harbor. A remarkable match between the populations clustering derived from analysis of environmental variables and the ordination of the populations according to species composition was found.
    
    \item The identification of differences, both in environmental conditions and floristic composition, between groups of oak woodland populations within the southern limit of their distribution, can help in the planning of forest management and restoration actions for these populations, particularly considering the importance of some environmental factors (\emph{e.g.} water availability) in key ecological aspects.
    
    \item A colonization process of \Qp into abandoned croplands in the Sierra Nevada mountain region has been observed despite the strong recruitment constraints described for this species. The melojo oak is being able to colonize abandoned areas within the same altitudinal level where it lives. Considering the climate change forecasts, it would be interesting to study the colonization capacity above its altitudinal limit. 
    
    \item The recolonization of abandoned mountain croplands by \Qp is conditioned by the  previous- and post-abandonment use of the croplands. In Sierra Nevada, differences have been observed in the recolonization of crops by the \Qp. Different management histories condition the abundance of regeneration of \Qp within abandoned crops. Thus, increased herbivory pressure after the abandonment of croplands seems to limit the expansion of the forest into mountain croplands.
    
    \item The \Qp forests of Sierra Nevada, like those of the rest of the Iberian Peninsula, have experienced an increase in total biomass in recent decades. The total biomass estimated for oak woodlands in the Sierra Nevada amounts to 9.94 Tg (1 Tg = 10$^{12}$ g), which represents a potential CO\textsubscript{2} sequestration of 17.33 Tg.
    
    \item The oak forests of Sierra Nevada, despite being located at the rear-edge of of their distribution, have a high potential for carbon sequestration. The estimated values of carbon stock for the Sierra Nevada \Qp forests are higher than those observed for other areas. These differences seem to be related to the relatively younger age of the \Qp trees of Sierra Nevada compared to other oak forests (due to the intense anthropic use to which they have been subjected). However, it would be necessary to carry out studies that analyze in depth the differences in the carbon sequestration capacity of this formation throughout its distribution range, and the factors that explain them. 
    
    \item There are differences in carbon sequestration potential among \Qp populations in the Sierra Nevada. Those populations that have been subjected to less anthropogenic disturbances have a greater structural richness, which is reflected in higher biomass values, and therefore, in a greater potential for carbon sequestration.
    
    \item The increase in temperature recorded in recent decades is causing changes in the dynamics of mountain ecosystems. Using an ontology system, the concurrence of changes in the snowfall pattern and productivity of Sierra Nevada oak forests has been observed. In areas where an earlier snowmelt date has been detected, a significant increase in summer primary productivity has also been observed for \Qp woodlands. This modification in water availability patterns due to climate change appears to be affecting the seasonal productivity of oak woodlands. This coupling between primary production and snow duration trends is most apparent for the western oak woodland populations of the Sierra Nevada, where 60\% of the pixels show an earlier snow melting date and an increase in summer productivity. 
    
    \item Melojo woodlands provide a large number of ecosystem services. In addition to the role of these forests as providers of regulating services (carbon sink) or provisioning services (use of their wood for wine aging), a large number of cultural ecosystem services provided by oak forests have been highlighted. The review of ecosystem services supplied by Sierra Nevada \Qp forests, combining expert criteria with local data, provides to natural resource managers with more information and tools to help them in the decision-making process (\emph{e.g}. conservation actions for the \Qp woodlands).
    
    \item \Qp forests located in the southern limit of its distribution area shown a great resilience to drought. Trees were highly sensitive to moisture availability, but both primary growth and secondary growth expressed high resilience to drought events over the short and the long term. The high values of resilience observed suggest that the Sierra Nevada \Qp populations are located in a geographical but not a climatic or ecological rear edge.
    
    \item A positive trend in primary productivity and secondary growth of Sierra Nevada \Qp woodlands has been observed in recent years. This could be related to a positive effect of warming for species found in cold-limited high elevation sites. It is also important to note that this response is different from the negative growth trend that would be expected for populations at the southern edge of their distribution, as has been observed for other temperate and Mediterranean species. 
    
    \item Resilience of Sierra Nevada \Qp woodlands to drought events was not spatially homogeneous across the mountain range, due to differences in ecological conditions and/or past management legacies. The large variability in responses among close neighboring sites, manifested by differences in tree growth, climatic sensitivity, and tree resilience, suggests that responses to drought are site-dependent and could drastically vary in extremely narrow spatial gradients, such as those occurring in mountainous regions. 

    \item The rear-edge concept should also consider historical aspects in addition to the geographic, climatic, and genetic ones, particularly in areas with a long history of human management, such as Mediterranean mountains.
    
\end{enumerate}