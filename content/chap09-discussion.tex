% !TEX root = ../my-thesis.tex
%
\chapter*{\textcolor{ctcolormain}{Discusión General}}\label{sec:discussions}
\newpage


--> PRIMERO. Que variables son las más importantes

La distribución de \Qp está condicionada por el periodo de sequía estival, y se ha apuntado que es necesario un mínimo de 100-150 mm de precipitación estival para su supervivencia \autocite{BlancoCastroetal2005BosquesIbericos,GarciaJimenez20099230Robledales}. Varios estudios bioclimáticos han revelado la importacia de los índices pluviométricos y ombrotérmicos en la separación entre bosques templados y mediterráneos \autocite{delRioetal2007BioclimaticAnalysis} para esta especie. A una escala más detallada, la distribución de esta especie se rige por un complejo gradiente relacionado con la temperatura, la precipitación y la radiación \autocite{Gavilanetal2007ModellingCurrent,Urbietaetal2011MediterraneanPine}. En el capítulo \ref{sec:multivar}, hemos encontrado que la separación entre las poblaciones de \Qp en Sierra Nevada, está  relacionada con el patrón espacial de precipitación existente para esta región de montaña \autocite{Pereiraetal2016SpatialInterpolation}. De hecho, la precipitación estival y anual son de los factores más importantes para explicar la distribución de los bosques de \Qp en Sierra Nevada. Los robledales situados al norte y al noroeste, se localizan en fondos de valle con y orientación norte, donde la humedad relativa es mayor como consecuencia de una menor radiación solar. Por otro lado, las poblaciones de la vertiente sur de Sierra Nevada reciben un aporte extra de agua procedente del aire húmedo del mar \autocite{MartinezParrasMoleroMesa1982EcologiaFitosociologia}. Las diferencias en la disponibilidad de agua entre las poblaciones de robles podrían afectar a varios procesos ecológicos como el crecimiento de los árboles \autocites[ver capítulo \ref{sec:dendro} ][]{GeaIzquierdoCanellas2014LocalClimate,PerezLuqueetal2020LanduseLegacies}, la germinación y supervivencia de las plántulas \autocites{Gomez2003ImpactVertebrate, GomezAparicioetal2008OakSeedling,Mendozaetal2009SeedingExperiment}, así como a la regeneración de la especie \autocites{Gomezetal2001ProblemasRegeneracion}, debido principalmente al papel clave de la disponibilidad de agua a escala de micrositio, que facilitan la germinación y el establecimiento de las plántulas.

--> SEGUNDO Hablar de la separación de grupos (heading: Ecological diversity within the rear-edge) 
Al explorar las características de las poblaciones de roble melojo en Sierra Nevada, hemos observado la existencia de grupos separados de poblaciones basándonos en sus características ambientales, siendo la radiación y la precipitación las principales variables discriminantes (ver capítulo \ref{sec:multivar}). Estas diferencias van en la misma línea de otros estudios, señalando una dinámica ecológica diferencial para estos bosques en Sierra Nevada. Así por ejemplo, se ha observado como la productividad primaria de estos ecosistemas (medida mediante el uso de teledetección) presenta un comportamiento espacial heterogéneo, con los robledales de la vertiente sur mostrando un mayor verdor anual de la vegetación que los de la vertiente norte \autocites[ver capítulo \ref{sec:onto} ][]{Dionisioetal2012SatelliteBasedMonitoring,PerezLuqueetal2015OntologicalSystem,PerezLuqueetal2020LanduseLegacies}. Asimismo, otros estudios señalan diferencias en la dinámica estacional del verdor \autocite{Dionisioetal2012SatelliteBasedMonitoring}, y en la tendencia temporal de la productividad primaria \autocites{PerezLuqueetal2015OntologicalSystem,AlcarazSeguraetal2016ChangesVegetation}, esto último parece estar relacionado con la diferente tendencia observada para la cubierta de nieve en las vertientes norte y sur de Sierra Nevada (ver capítulo \ref{sec:onto}).  

Curiosamente, también hemos encontrado diferencias en la diversidad de especies entre los grupos de poblaciones derivadas de la agrupación basada en variables ambientales (ver capítulo \ref{sec:multivar}). Estos resultados son consistentes con los aportados por \textcite{Loriteetal2008PhytosociologicalReview}, que señalaron que las diferencias observadas para el componente florístico en las poblaciones de \Qp Sierra Nevada están relacionadas con las condiciones microclimáticas. En este sentido, los robledales situados en el norte de Sierra Nevada presentan mayor similitud florística con los situados en el centro de la distribución de \Qp que con los que se localizan en la vertiente sur de Sierra Nevada (geográficamente más cercanos) \autocites{Loriteetal2008PhytosociologicalReview}. 

Además de los factores climáticos, las diferencias florísticas entre las poblaciones de roble de Sierra Nevada también podrían estar relacionadas con el impacto antropogénico sufrido, ya que las perturbaciones antrópicas pueden afectar a los patrones florísticos, como se ha documentado para los robledales del centro de la Península Ibérica \autocite{Gavilanetal2000EffectsDisturbance}. Por ejemplo, el robledal del Camarate (que representa el grupo de población Norte en Sierra Nevada, ver capítulo \ref{sec:multivar}) mostró una mayor diversidad y riqueza de especies vegetales, lo que puede estar relacionado con un mejor estado de conservación, ya que esta población ha estado menos expuesta a la intensa actividad antropogénica \autocite{JimenezOlivencia1991PaisajesSierra}. Por el contrario, las poblaciones de robledal situadas en la vertiente sur, mostraron una composición florística más pobre condicionada tanto por el clima como por el intenso uso del suelo \autocite{CamachoOlmedoetal2002DinamicaEvolutiva,AlAallalietal1998EstudioVegetacion}. Por tanto, podemos atrevernos a afirmar la importancia del uso del suelo para el funcionamiento de las poblaciones dentro del borde posterior de su distribución. 

La notable coincidencia entre la agrupación de las poblaciones derivada del análisis de las variables ambientales y la ordenación de las poblaciones según la composición de especies, sugieren una relación entre la heterogeneidad de los factores ambientales y la variabilidad de la composición de especies para estos bosques. Esta diversidad de condiciones ecológicas para las poblaciones de \Qp situadas en este borde posterior, está en consonancia con los altos niveles de diversidad genética mostrados por las poblaciones de este roble en Sierra Nevada \autocites{ValbuenaCarabanaGil2013GeneticResilience, ValbuenaCarabanaGil2017CentenaryCoppicing}. La heterogeneidad climática y topográfica que existe en Sierra Nevada ofrece una gran diversidad de microhábitats, lo que ha permitido que esta región montañosa actúe como refugio de diferentes especies \autocites{MedailDiadema2009GlacialRefugia, GomezLunt2007RefugiaRefugia,BlancoPastoretal2019TopographyExplains}, incluso para las especies de \emph{Quercus} caducifolios durante el último periodo glacial \autocites{Breweretal2002SpreadDeciduous, Olaldeetal2002WhiteOaks,RodriguezSanchezetal2010TreeRange}. De hecho, existen evidencias fósiles y genéticas que sugieren que diferentes especies de \emph{Quercus} sólo sobrevivieron en refugios del sur durante el último máximo glacial \autocite{Breweretal2002SpreadDeciduous,Petitetal2002IdentificationRefugia,BhagwatWillis2008SpeciesPersistence,BirksWillis2008AlpinesTrees}. La persistencia en un refugio sugiere una combinación de un entorno local moderadamente adecuado que amortigua el clima regional, y una relativa tolerancia al cambio climático, ya sea por una pronunciada plasticidad fenotípica, y/o por la capacidad de adaptación \autocites{Gavinetal2014ClimateRefugia}. Este podría ser muy bien el caso de \Qp, una especie que alberga una alta diversidad genética \autocite{ValbuenaCarabanaGil2013GeneticResilience}, situada en una región montañosa con una topografía compleja que podría proteger a las poblaciones locales contra los rápidos cambios climáticos y permitir que las especies persistan a pesar de los entornos regionales desfavorables.


--> TERCERO Implicaciones de este agrupamiento para modelización y para el resto de la tesis
--> Ver apartado Implications for forecasting and modelling 

--> Importancia del uso del suelo: 
- procesos de colonización de cultivo (chap4)
- densificación de las masas (seq carbono)

* Valores estimados de biomasa  superiores en SN que en el resto de su distribucion (142.27 -157.13 \mgha 

- afecta al crecimiento (dendro)

### Cambio en el clima: 

1. Varios trabajos han señalado un aumento de las temperaturas y una disminución de la precipitación en el área Mediterránea \autocites{GarciaRuizetal2011MediterraneanWater,GiorgiLionello2008ClimateChange}. 

En la región Mediterránea se ha registrado en las últimas décadas un aumento generalizado de las temperaturas, así como un cambio en los patrones de precipitación \autocites{PerezBoscolo2010ClimateSpain,GiorgiLionello2008ClimateChange,Crameretal2020ClimateEnvironmental}. Para Sierra Nevada, usando datos datos de estaciones meteorológicas y mapas climáticos de alta resolución \autocites{Benitoetal2014ClimateSimulations}, también se han encontrado tendencias positivas para la temperaturas mínimas y máximas anuales, así como un patrón generalizado de reducción de la precipitación anual desde la década de 1960 \autocites{PerezLuqueetal2016SenalesCambio,PerezLuqueetal2021ClimaNevadaBase}. 


Esos cambios observados para las variables climáticas (ver Pérez-luque et al. ...) parecen afectar a varios procesos, como por ejemplo la dinámica de la cubierta de nieve en ambientes de montaña (Citas trujillo et al.; buscar más). En algunas montañas europeas se han registrado descensos significativos en la extensión y duración de la cubierta de nieve \autocite{Marty2008RegimeShift,MorenoRodriguezetal2005EvaluacionPreliminar,Nikolovaetal2013ChangesSnowfall,Scherreretal2004TrendsSwiss}. Para Sierra Nevada, se han observado además tendencias hacia un retraso en la fecha de inicio de la presencia de nieve, y hacia un adelanto en la fecha de fusión de la nieve \autocites{PerezLuqueetal2016SenalesCambio}. Tal y como hemos mostrado en el capítulo ONTO, las zonas por encima de los robledales 

Asimismo algunos trabajos han mostrado cambios tanto en

tal y como hemos mostrado en el CAPITU resultados obtenidos (ver capitulo ontologias), muestran tendencias hacias un adelanto de la fecha de fusión de la nieve. 



Consecuencias 

1.a Afección a la cubierta de nieve: 

 Se espera un adelanto en la fecha de fusión de la nieve, y que esta ocurra más rapído (INCLUIR DATOS DE SN) 

En nuestro caso, hemos observado que  

Así, encontramos que el 70\% de los pixeles que cubren los robledales, muestran tendencias hacia una fecha de deshielo mas temprana (esto es ... significativas y negativas en, ver 



1.b Aumento en la productividad anual y estacional
- Tendencia en el verdor anual (EVI, ontologia)


- Tendencia en el verdor de verano en NDVI de los robledales.

Mediante el uso del sistemas de ontologías (ver capítulo \ref{sec:onto}), hemos observado tendencias temporales siginificativas para el NDVI de verano en los robledales de Sierra Nevada. En concreto, el 75\% de los píxeles que cubren los robledales han mostrado tendencias positivas significativas, de las cuales un 80\% fueron fuertes o muy fuertes (CIFRAS). 



Conclusiónnes: 
* Aumento de la productividad 






- ndvi 
- patrón temporal ndvi + nieve 

Mediante el uso del sistemas de ontologías (ver capítulo \ref{sec:onto}), hemos observado un tendencias temporales siginificativas tanto para el NDVI de verano como para diversos parámetros relacionados con la cubierta de nieve. 




Probamos el sistema ontológico en un estudio de caso centrado en el hábitat del \Qp en Sierra Nevada. Identificamos tendencias significativas en el NDVI de verano para el 75\% de los píxeles cubiertos por el hábitat objetivo. Estos píxeles estaban localizados principalmente en parches de cara al norte (el aspecto se calculó utilizando el DEM). Estos resultados podrían explicarse por un patrón diferente de productividad estival entre los parches \Qp. También hemos descrito tendencias similares en los patrones de nieve: El 70\% de los píxeles muestran una tendencia significativa y negativa hacia una fecha de deshielo más temprana. La mayoría de esos píxeles también se encuentran en parches orientados al norte. Este resultado podría tener varias implicaciones hidrológicas y ecológicas: a) el agua de la nieve derretida está disponible para la vegetación más temprano cada año, lo que podría ayudar a los árboles caducifolios a superar la sequía estival, b) el suelo está libre de nieve durante un periodo más largo cada año, lo que podría proporcionar un área extra a las comunidades arbóreas para los desplazamientos altitudinales.

La ontología también ha ayudado a desvelar la co-ocurrencia de tendencias significativas tanto en la cobertura de nieve (factor abiótico) como en el funcionamiento del ecosistema (NDVI). Así, las manchas occidentales presentan un alto porcentaje de píxeles que muestran esta co-ocurrencia. Las implicaciones ecológicas de esta co-ocurrencia pueden explicarse argumentando que el deshielo más temprano proporciona agua a los árboles de \emph{Q. pyrenaica} cuando están en la mitad de su temporada de crecimiento. Este suministro de agua más temprano favorece que los árboles sean más productivos en verano. Por otro lado, las manchas del sur también muestran esta co-ocurrencia en sentido contrario: La falta de tendencias significativas en la productividad de verano para las manchas del sur podría explicarse por la falta de píxeles con tendencias hacia un deshielo más temprano en estas zonas. Aunque estos resultados son todavía preliminares, hemos establecido un vínculo entre el estado de un factor abiótico y el funcionamiento de los ecosistemas. Algunas actividades forestales pueden programarse en función de las tendencias observadas. Podría ser útil, por ejemplo, reforzar las manchas occidentales plantando árboles \Qp. Estos nuevos árboles podrían aprovechar los veranos productivos para crear bosques más densos. Estos resultados ecológicos son similares a otros encontrados en diferentes hábitats (Trujillo et al.~2012).

Los resultados (tanto ecológicos como metodológicos) demuestran que la información de las series temporales de MODIS es útil para evaluar el funcionamiento de un hábitat terrestre de Natura 2000. Hemos descrito el comportamiento temporal de los bosques de \Qp en Sierra Nevada, distinguiendo entre parches situados en zonas con diferentes condiciones ambientales. También hemos mostrado las tendencias temporales de varios indicadores de funcionamiento. Las tendencias descubiertas podrían ayudar a los gestores a evaluar el estado de conservación de este hábitat. También pueden construir planes de gestión utilizando el conocimiento proporcionado por nuestra ontología (para decidir dónde ubicar las plantaciones teniendo en cuenta las tendencias de productividad). También hemos descrito el comportamiento de un factor abiótico clave: la cubierta de nieve; y hemos calculado las tendencias de varios indicadores relacionados con la cubierta de nieve (duración de la nieve, fecha de fusión de la cubierta de nieve, etc.). Esto podría ayudar a los gestores a identificar los lugares en los que las tendencias de la cubierta de nieve podrían cambiar en los próximos años. Por último, hemos detectado relaciones entre las tendencias de la productividad del hábitat y la fecha de fusión del manto de nieve para el hábitat objetivo. Todo este conocimiento se ofrece a los usuarios (principalmente gestores y científicos) a través de un portal web, cuyo uso no requiere conocimientos de teledetección. Así pues, creemos que este trabajo es un valioso ejemplo de sistema experto basado en la web y creado con un enfoque interdisciplinar.


- sequías -> resiliencia 

En Sierra Nevada, al igual que en otras zonas del sur de Europa, se ha observado en los últimos años un aumento en la duración, frecuencia y severidad de los eventos de sequía \autocites[ver capítulo \ref{sec:dendro} y apéndice \ref{sec:appendix:dendro};][]{VicenteSerranoetal2014EvidenceIncreasing,Staggeetal2017ObservedDrought,Spinonietal2015EuropeanDrought,Pascoaetal2017DroughtTrends}, con una tendencia hacia veranos más secos \autocites{Spinonietal2017PanEuropeanSeasonal} (\figreft{fig:intro:sequia}). Este aspecto es especialmente importante para las poblaciones de \Qp que tienen centrado su máximo de crecimiento ... 

El incremento en la frecuencia y severidad de las sequías está alterando el funcionamiento de los ecosistemas mediterráneos a diferentes escalas \autocites{Penuelasetal2017ImpactsGlobal,Forneretal2018ExtremeDroughts,Liuetal2020EffectsDecadal,OgayaPenuelas2021ClimateChange}, puesto que la sequía afecta a aspectos fisiológicos, funcionales, estructurales y demográficos de los ecosistemas forestales \autocites{Allenetal2010GlobalOverview, Assaletal2016SpatialTemporal}. No obstante, se están observando respuestas divergentes de los ecosistemas forestales a la sequía \autocites{Andereggetal2020DivergentForest}, poniendo de manifiesto la importancia de otros aspectos como el momento en el que ocurre la sequía \autocites{Huangetal2018DroughtTiming}. Esto es de especial relevancia para especies de frondosas como el \Qpy que presenta una fenología de crecimiento bien marcada \autocites{PerezdeLisetal2016ChangesSpring}. 

A pesar de ello ... 


El aumento de los eventos extremos, tales como sequías, tormentas, inundaciones, etc. \autocite{IPCC2013ClimateChange}. A pesar de que la sequía es una característica del clima mediterráneo \autocites{Lionello2012}, en las últimas décadas se ha registrado un incremento en la duración, frecuencia y severidad de los eventos de sequía \autocites{LloydHughesSaunders2002DroughtClimatology, Sousaetal2011TrendsExtremes,Colletal2017DroughtVariability}, particularmente en el sur de Europa \autocites{VicenteSerranoetal2014EvidenceIncreasing,Staggeetal2017ObservedDrought,Spinonietal2015EuropeanDrought,Pascoaetal2017DroughtTrends}, donde además se ha observado una tendencia hacia veranos más secos \autocites{Spinonietal2017PanEuropeanSeasonal} (\figreft{fig:intro:sequia}). Este hecho cobra especial relevancia para el área Mediterránea, considerada una de las más vulnerables frente al cambio climático \autocites{Giorgi2006ClimateChange}, ya que las proyecciones a futuro pronostican un aumento de la severidad de los eventos climáticos extremos extremos \autocites{Hoerlingetal2012IncreasedFrequency,IPCC2013ClimateChange,Trenberthetal2014GlobalWarming,Spinonietal2018WillDrought}.  

Los eventos de sequía severa afectan a la dinámica de crecimiento de \Qp. Los resultados del capítulo \ref{sec:dendro} muestran claramente una reducción en el crecimiento primario (verdor medido usando EVI) y en el crecimiento secundario (BAI) para los eventos de sequía de 2005 y 2012. Asimismo, observamos que durante los eventos de sequía extrema, como la registrada en 1995, se produjo la mayor reducción del crecimiento radial. Ese evento de sequía causó graves y extensos daños en los ecosistemas Mediterráneos de la Península Ibérica \autocite{Penuelasetal2001SevereDrought,Gazoletal2018ForestResilience}.

