% !TEX root = ../my-thesis.tex
%
\selectlanguage{spanish}
\chapter{\textcolor{ctcolormain}{Discusión General}}\label{sec:discussions}
\newpage

\subsection*{Diversidad ecológica dentro del borde posterior de distribución de \Qp. }\label{sec:discussions:diversidad}

La distribución de \Qp está condicionada por el periodo de sequía estival, necesitando un mínimo de 100-150 mm de precipitación estival para su supervivencia \autocite{BlancoCastroetal2005BosquesIbericos,GarciaJimenez20099230Robledales}. Varios autores han señalado la importancia de los índices pluviométricos y ombrotérmicos en la separación entre bosques templados y mediterráneos para esta especie a lo largo de su área de distribución \autocite{delRioetal2007BioclimaticAnalysis}. A escalas más detalladas, la distribución de esta especie se rige por un complejo gradiente relacionado con la temperatura, la precipitación y la radiación \autocites{Gavilanetal2007ModellingCurrent,Urbietaetal2011MediterraneanPine}. En el capítulo \ref{sec:multivar}, hemos encontrado que la precipitación estival y anual son factores muy importantes para explicar la distribución de los bosques de \Qp en Sierra Nevada. De hecho, hemos observado diferentes grupos de poblaciones de \Qp en esta región montañosa discriminados principalmente por la precipitación y la radiación. Los robledales situados al norte y al noroeste, se localizan en fondos de valle con orientaciones norte, donde la humedad relativa es mayor como consecuencia de una menor radiación solar. Por otro lado, las poblaciones de la vertiente sur de Sierra Nevada reciben un aporte extra de agua procedente del aire húmedo del mar \autocite{MartinezParrasMoleroMesa1982EcologiaFitosociologia}. Las diferencias en la disponibilidad de agua entre las poblaciones de robles podrían afectar a varios procesos ecológicos como la germinación y supervivencia de las plántulas \autocites{Gomez2003ImpactVertebrate, GomezAparicioetal2008OakSeedling,Mendozaetal2009SeedingExperiment}, así como a la regeneración de la especie \autocites{Gomezetal2001ProblemasRegeneracion}, debido principalmente al papel clave de la disponibilidad de agua a escala de micrositio, que facilitan la germinación y el establecimiento de las plántulas. Asimismo, como se ha observado para otras especies de \emph{Quercus} \autocites[\emph{e.g.}][]{Tessieretal1994DeciduousQuercus,DiFilippoetal2010ClimateChange,GeaIzquierdoetal2011TreeringsReflect,GarciaGonzalezSoutoHerrero2017EarlywoodVessel}, la disponibilidad de agua es uno de los principales factores limitantes que afectan al crecimiento secundario de \Qp (ver resultados obtenidos en el capítulo \ref{sec:dendro}). 

Las poblaciones de \Qp situadas en Sierra Nevada, que representan un borde posterior (\emph{rear-edge}), no son ecológicamente homogéneas, ni en sus condiciones ambientales ni en su composición de especies. Al explorar las características de las poblaciones de roble melojo en Sierra Nevada, hemos observado la existencia de grupos separados de poblaciones basándonos en sus características ambientales, siendo la radiación y la precipitación las principales variables discriminantes (ver capítulo \ref{sec:multivar}). Esta heterogeneidad va en la línea de otros estudios que señalan una dinámica ecológica diferencial para estos bosques en Sierra Nevada. Así por ejemplo, la productividad primaria de estos ecosistemas (medida mediante el uso de teledetección) difiere entre las poblaciones de \Qp de Sierra Nevada, con los robledales de la vertiente sur mostrando un mayor verdor anual de la vegetación que los de la vertiente norte \autocites[ver capítulo \ref{sec:onto} ][]{Dionisioetal2012SatelliteBasedMonitoring,PerezLuqueetal2015OntologicalSystem,PerezLuqueetal2020LanduseLegacies}. Asimismo, otros estudios encontraron diferencias en la dinámica estacional del verdor \autocite{Dionisioetal2012SatelliteBasedMonitoring}, y en la tendencia temporal de la productividad primaria \autocites{PerezLuqueetal2015OntologicalSystem,AlcarazSeguraetal2016ChangesVegetation}, esto último parece estar relacionado con la diferente tendencia observada para la cubierta de nieve en las vertientes norte y sur de Sierra Nevada (ver capítulo \ref{sec:onto}).  

Curiosamente, también hemos encontrado diferencias en la diversidad de especies entre los grupos de poblaciones derivadas de la agrupación basada en variables ambientales (ver capítulo \ref{sec:multivar}). Estos resultados son consistentes con los aportados por \textcite{Loriteetal2008PhytosociologicalReview}, que señalaron que las diferencias observadas para el componente florístico en las poblaciones de \Qp de Sierra Nevada están relacionadas con las condiciones microclimáticas. En este sentido, los robledales situados en el norte de Sierra Nevada presentan mayor similitud florística con los situados en el centro de la distribución de \Qp que con los que se localizan en la vertiente sur de Sierra Nevada (geográficamente más cercanos) \autocites{Loriteetal2008PhytosociologicalReview}. Además de los factores climáticos, las diferencias florísticas entre las poblaciones de \Qp en Sierra Nevada también podrían estar relacionadas con el impacto antropogénico sufrido, ya que las perturbaciones antrópicas pueden afectar a los patrones florísticos, como se ha documentado para los robledales del centro de la Península Ibérica \autocite{Gavilanetal2000EffectsDisturbance}. Por ejemplo, el robledal del Camarate (que representa el grupo de población Norte en Sierra Nevada, ver capítulo \ref{sec:multivar}) mostró una mayor diversidad y riqueza de especies vegetales, lo que puede estar relacionado con un mejor estado de conservación, ya que esta población ha estado menos expuesta a la intensa actividad antropogénica \autocite{JimenezOlivencia1991PaisajesSierra}. Por el contrario, las poblaciones de robledal situadas en la vertiente sur, mostraron una composición florística más pobre condicionada tanto por el clima como por el intenso uso del suelo \autocite{CamachoOlmedoetal2002DinamicaEvolutiva,AlAallalietal1998EstudioVegetacion}. Por tanto, podemos atrevernos a afirmar la importancia del uso del suelo para entender el estado actual de los robledales de \Qp dentro del borde posterior de su distribución. 

La notable coincidencia entre la agrupación de las poblaciones derivada del análisis de las variables ambientales y la ordenación de las poblaciones según la composición de especies, sugieren una relación entre la heterogeneidad de los factores ambientales y la variabilidad de la composición de especies para estos bosques. Esta diversidad de condiciones ecológicas para las poblaciones de \Qp situadas en este borde posterior, está en consonancia con los altos niveles de diversidad genética mostrados por las poblaciones de esta especie en Sierra Nevada \autocites{ValbuenaCarabanaGil2013GeneticResilience, ValbuenaCarabanaGil2017CentenaryCoppicing}. La heterogeneidad climática y topográfica que existe en Sierra Nevada ofrece una gran diversidad de microhábitats, lo que ha permitido que esta región montañosa actúe como refugio de diferentes especies \autocites{MedailDiadema2009GlacialRefugia, GomezLunt2007RefugiaRefugia,BlancoPastoretal2019TopographyExplains}, incluso para las especies de \emph{Quercus} caducifolios durante el último periodo glacial \autocites{Breweretal2002SpreadDeciduous, Olaldeetal2002WhiteOaks,RodriguezSanchezetal2010TreeRange}. De hecho, existen evidencias fósiles y genéticas que sugieren que diferentes especies de \emph{Quercus} sólo sobrevivieron en refugios del sur durante el último máximo glacial \autocite{Breweretal2002SpreadDeciduous,Petitetal2002IdentificationRefugia,BhagwatWillis2008SpeciesPersistence,BirksWillis2008AlpinesTrees}. La persistencia en un refugio sugiere una combinación de un entorno local moderadamente adecuado que amortigua el clima regional, y una relativa tolerancia al cambio climático, ya sea por una pronunciada plasticidad fenotípica, y/o por la capacidad de adaptación \autocites{Gavinetal2014ClimateRefugia}. Este podría ser muy bien el caso de \Qp, una especie que alberga una alta diversidad genética \autocite{ValbuenaCarabanaGil2013GeneticResilience}, situada en una región montañosa con una topografía compleja que podría proteger a las poblaciones locales contra los rápidos cambios climáticos y permitir que las especies persistan a pesar de los entornos regionales desfavorables.

La identificación de diferentes grupos de poblaciones en función de las variables ambientales a escala de detalle (ver capítulo \ref{sec:multivar}), es importante a la hora de modelizar la distribución de una especie y prever los impactos del cambio global sobre ella, ya que los factores que controlan la distribución de las especies pueden variar en función de la escala de observación \autocites{GuisanThuiller2005PredictingSpecies,Urbietaetal2008SoilWater,SanchezdeDiosetal2009PresentFuture}. Los resultados que hemos obtenido en el capítulo \ref{sec:multivar}, que indican una separación de poblaciones de \Qp en función de la disponibilidad de agua, sugieren la necesidad de incorporar estos resultados en los modelos de predicción del impacto de cambio climático sobre esta especie.  Incorporar las adaptaciones locales de las poblaciones en los modelos predictivos, puede ayudar a evitar representaciones erróneas del desplazamiento del área de distribución de las especies bajo condiciones climáticas cambiantes \autocite{BenitoGarzonetal2011IntraspecificVariability}. Esto es especialmente importante en el caso de las especies con bordes posteriores situados en regiones de montaña, ya que estas zonas ofrecen una amplia diversidad de microhábitats debido a la heterogeneidad climática y topográfica \autocite{MedailDiadema2009GlacialRefugia}.
Por ejemplo, algunos trabajos recientes han realizado modelos de alta resolución de la distribución de especies arbóreas relictas en montañas Mediterráneas del sur (\emph{e.g}. \emph{Abies pinsapo, Pinus sylvetris} y \emph{P. nigra}) proporcionando información útil para las acciones de gestión forestal \autocite{LopezTiradoHidalgo2014HighResolution}.

\subsection*{Los robledales de \Qp en Sierra Nevada frente al cambio climático
}\label{sec:discussions:climate}

Varios son los trabajos que han documentando un aumento generalizado de las temperaturas, así como un cambio en los patrones de precipitación para la región Mediterránea \autocites[\emph{e.g.}][]{PerezBoscolo2010ClimateSpain,GarciaRuizetal2011MediterraneanWater,GiorgiLionello2008ClimateChange,Crameretal2020ClimateEnvironmental}. Estos patrones generales también han sido observados para Sierra Nevada, donde varios estudios, a partir de datos de estaciones meteorológicas y usando mapas climáticos de alta resolución, han encontrado tendencias positivas para las temperaturas mínimas y máximas anuales, así como un descenso generalizado en la precipitación desde 1960 \autocite{Benitoetal2014ClimateSimulations,PerezLuqueetal2016SenalesCambio,PerezLuqueetal2021ClimaNevadaBase,PerezPalazonetal2015ExtremeValues}. Esos cambios observados en las variables climáticas parecen afectar a varios procesos ecológicos, como por ejemplo la dinámica de la cubierta de nieve en ambientes de montaña \autocite{Trujilloetal2012ElevationdependentInfluence}. Así, en algunas montañas europeas se han registrado descensos significativos en la extensión y duración de la cubierta de nieve \autocite{Marty2008RegimeShift,MorenoRodriguezetal2005EvaluacionPreliminar,Nikolovaetal2013ChangesSnowfall,Scherreretal2004TrendsSwiss}. En Sierra Nevada, además de estos cambios, se han observado modificaciones en la fecha de inicio y de fusión de la cubierta de nieve \autocites{PerezLuqueetal2016SenalesCambio}. Específicamente se han documentado para los últimos 20 años un retraso en la fecha de inicio de la presencia de nieve, y un adelanto en la fecha de fusión de la nieve \autocites{PerezLuqueetal2016SenalesCambio}. Por ejemplo, para el 70\% de los píxeles que coinciden con la distribución de los robledales en Sierra Nevada, hemos encontrado tendencias hacia una fecha de deshielo mas temprana (ver capítulo \ref{sec:onto}). 

Las alteraciones en los patrones de disponibilidad hídrica, tal y como apuntábamos en los objetivos de esta memoria doctoral, pueden afectar al funcionamiento ecosistémico de los robledales en Sierra Nevada. Mediante el uso del sistema de ontologías (ver capítulo \ref{sec:onto}), hemos observado tendencias temporales significativas para el NDVI de verano en los robledales. En concreto, el 75\% de los píxeles que cubren los robledales han mostrado tendencias positivas significativas. El sistema de ontologías desarrollado, nos ha ayudado a desvelar la co-ocurrencia de tendencias significativas tanto en la cobertura de nieve (factor abiótico) como en el funcionamiento del ecosistema (NDVI). Por ejemplo, en las poblaciones de robledal situadas en la parte más occidental de Sierra Nevada, la mayor parte de sus píxeles muestran esta co-ocurrencia. Las implicaciones ecológicas de esta co-ocurrencia pueden explicarse argumentando que el deshielo más temprano proporciona agua a los árboles de \emph{Q. pyrenaica} cuando están en la mitad de su temporada de crecimiento. Este suministro de agua más temprano puede favorecer que los árboles sean más productivos en verano. Por otro lado, las poblaciones de robledal situadas en el sur de Sierra Nevada, también muestran esta co-ocurrencia en sentido contrario: la falta de tendencias significativas en la productividad de verano para ésta poblaciones, podría explicarse por la falta de píxeles con tendencias hacia un deshielo más temprano en estas zonas. Aunque estos resultados hay que considerarlos con cautela, parecen indicar la existencia de un vínculo entre el estado de un factor abiótico (duración de la cubierta de nieve) y el funcionamiento de los ecosistemas (productividad), tal y como ha sido documentado en otras regiones montañosas \autocite{Wanetal2014ChangeSnow,DyeTucker2003SeasonalityTrends}. Estos resultados pueden utilizarse en la programación de algunas actuaciones forestales de mejora de los robledales. Por ejemplo, si entre las actuaciones forestales se contemplara la realización de refuerzos poblacionales mediante plantación, éstas se pueden realizar en aquellas poblaciones donde se haya observado un patrón de mayor disponibilidad de agua, lo cual aumentaría la posibilidad de éxito de la actuación, teniendo en cuenta la importancia de la disponibilidad de agua para esta especie (ver capítulo \ref{sec:multivar}). 

En Sierra Nevada, al igual que en otras zonas del sur de Europa, se ha observado en los últimos años un aumento en la duración, frecuencia y severidad de los eventos de sequía \autocites[ver capítulo \ref{sec:dendro} y apéndice \ref{sec:appendix:dendro};][]{VicenteSerranoetal2014EvidenceIncreasing,Staggeetal2017ObservedDrought,Spinonietal2015EuropeanDrought,Pascoaetal2017DroughtTrends}, con una tendencia hacia veranos más secos \autocites{Spinonietal2017PanEuropeanSeasonal}. Esta situación está alterando el funcionamiento de los ecosistemas mediterráneos a diferentes escalas \autocites{Penuelasetal2017ImpactsGlobal,Forneretal2018ExtremeDroughts,Liuetal2020EffectsDecadal,OgayaPenuelas2021ClimateChange}, puesto que la sequía afecta a aspectos fisiológicos, funcionales, estructurales y demográficos de los ecosistemas forestales \autocites{Allenetal2010GlobalOverview, Assaletal2016SpatialTemporal}. No obstante, se están observando respuestas divergentes de los ecosistemas forestales a la sequía \autocites{Andereggetal2020DivergentForest}, poniendo de manifiesto la importancia de otros aspectos, como por ejemplo, el momento en el que ocurre la sequía \autocites{Huangetal2018DroughtTiming}. Esto es de especial relevancia para especies como \Qp que presenta una fenología de crecimiento bien marcada con su máximo de crecimiento en final de primavera y verano \autocites{PerezdeLisetal2016ChangesSpring}. 

Los eventos de sequía severa afectan a la dinámica de crecimiento de \Qp. Los resultados del capítulo \ref{sec:dendro} muestran claramente una reducción en el crecimiento primario (verdor medido usando EVI) y en el crecimiento secundario (BAI) para los eventos de sequía de 2005 y 2012. Asimismo, observamos que durante los eventos de sequía extrema, como la registrada en 1995, se produjo la mayor reducción del crecimiento radial. Ese evento de sequía causó graves y extensos daños en los ecosistemas Mediterráneos de la Península Ibérica \autocite{Penuelasetal2001SevereDrought,Gazoletal2018ForestResilience}. 

Las respuestas de los árboles a la sequía son dependientes del sitio \autocite{Babstetal2013SiteSpecies}, particularmente para las poblaciones situadas en su borde posterior de distribución \autocites{CavinJump2017HighestDrought,DoradoLinanetal2017LargescaleAtmospheric}. 
Tal y como hemos observado en el capítulo \ref{sec:dendro}, tanto el verdor de la vegetación (EVI), como el crecimiento secundario de los árboles (BAI), se vieron más afectados por los eventos de sequía en las poblaciones de robledal situadas en la cara norte de Sierra Nevada que las situadas en la cara sur. Específicamente, encontramos que las poblaciones situadas en la cara norte mostraron anomalías de EVI más negativas para la sequía de 2005 que las situadas en la cara sur. Asimismo, las correlaciones entre el crecimiento secundario de los árboles y el índice SPEI (tanto del año hidrológico como de verano) fueron más intensas para las poblaciones de la cara norte, lo cual se puede interpretar como una mayor sensibilidad a la sequía en los sitios más secos \autocite{GeaIzquierdoCanellas2014LocalClimate}. Por otro lado, es interesante destacar la gran variabilidad en respuesta al clima mostrada por las poblaciones de roble a lo largo de un estrecho gradiente. Por ejemplo, los árboles situados en las zonas altas del robledal de Cáñar mostraron un BAI superior a los situados en elevaciones más bajas, aunque ambos sitios de muestreo se encuentren muy cerca el uno del otro. 

No obstante, a pesar de los severos eventos de sequía de las últimas décadas, llama poderosamente la atención la tendencia positiva para el verdor de la vegetación mostrada por los bosques de \Qp durante los últimos 20 años (ver resultados de los capítulos \ref{sec:dendro} y \ref{sec:onto}). De igual modo, para el crecimiento de los árboles también observamos tendencias positivas en la última década para los robledales de Sierra Nevada. Estos resultados concuerdan con las tendencias a largo plazo exhibidas por esta especie en lugares húmedos y fríos a lo largo de su área de distribución \autocite{GeaIzquierdoCanellas2014LocalClimate}. Esto podría estar relacionado con un efecto positivo no lineal del aumento de temperaturas, para las especies situadas en sitios a mayor altitud con crecimiento limitado por el frío \autocites{Salzeretal2009RecentUnprecedented,GeaIzquierdoCanellas2014LocalClimate}. 

Un aspecto importante a destacar es que, para las poblaciones situadas en los bordes posteriores de distribución que se encuentran amenazadas por el cambio climático, se esperaban tendencias de crecimiento negativas, tal y como se ha mostrado para algunas especies templadas y mediterráneas \autocites{SanchezSalgueroetal2012DroughtMain,Camareroetal2015NotEarly,DoradoLinanetal2017CoexistenceMediterraneanTemperate}. Sin embargo, nuestros resultados (ver capítulos \ref{sec:dendro} y \ref{sec:onto}), indican que las poblaciones de robledal en Sierra Nevada, situadas en uno de los bordes posteriores de distribución, mostraron tendencias positivas para el EVI (crecimiento primario) y en el BAI de los árboles (crecimiento secundario). 

\subsection*{Los robledales de \Qp en Sierra Nevada frente a los cambios de uso 
}\label{sec:discussions:uso}

Los cambios de uso del suelo se consideran los principales impulsores del cambio global \autocites{Butchartetal2010GlobalBiodiversity,Winkleretal2021GlobalLand}, afectando a la biodiversidad \autocites{Sala2000GlobalBiodiversity}, modificando diferentes procesos ecológicos \autocites{Lindenmayeretal2012LandUse} y alterando la provisión de servicios ecosistémicos \autocites{Hasanetal2020ImpactLand}. Los bosques de \Qp, al igual que otras formaciones forestales de la región Mediterránea, se han visto sometidos a intensas presiones antropogénicas a lo largo del tiempo \autocites{GarciaJimenez20099230Robledales, AlbaSanchezetal2021EarlyAnthropogenic}, que han llevado a la reducción de su área de distribución, así como a la modificación de sus patrones florísticos y estructurales \autocites{Gavilanetal2000EffectsDisturbance,Calvoetal1999PostfireSuccession,Tarregaetal2006ForestStructure}. Históricamente, los bosques de \Qp han sido explotados principalmente para la obtención de leña, carbón vegetal y taninos \autocites{RuizdelaTorre2006FloraMayor,SanchezPalomaresetal2008EstacionesEcologicas}. Algunas zonas se quemaron y aclararon para crear pastos con bajas densidades de árboles maduros que proporcionaban bellotas, leña y grandes áreas para el pastoreo \autocites{HerreraCalvo2016UsoPastoral,Alvarezetal2009CambiosEstructura,ValbuenaCarabanaGil2017CentenaryCoppicing}. Todos estos procesos antropogénicos han transformado los robledales de una manera tan profunda que es difícil encontrar rodales que puedan considerarse bosques naturales \autocites{RuizdelaTorre2006FloraMayor}. 

La revisión de diversos documentos históricos (ver capítulos \ref{sec:coloniza} y \ref{sec:dendro}) ha revelado que las cortas, la extracción de leña, la producción de carbón vegetal y la minería, entre otras actividades antropogénicas, han afectado intensamente a los bosques de Sierra Nevada. De hecho, estudios previos han estimado una pérdida histórica de en torno al 90\% de la cobertura arbórea para especies caducifolias de \emph{Quercus} en elevaciones medias y bajas de Sierra Nevada \autocite{JimenezOlivenciaetal2015MedioSiglo}. 
Las diversas estructuras forestales que encontramos en la actualidad en las poblaciones de \Qp de Sierra Nevada (ver resultados en capítulos \ref{sec:multivar}, \ref{sec:coloniza}, \ref{sec:carbon} y \ref{sec:dendro}) parecen ser fruto de intensas y diferentes historias de uso y manejo que han sufrido estas formaciones forestales (\figreft{fig:carbon:schema}). En la vertiente norte de Sierra Nevada (por ejemplo el robledal de San Juan, GEN) los usos del suelo se han distribuido históricamente a lo largo de un gradiente de elevación: pastizales y matorrales para la ganadería en las cotas más altas; a continuación, masas forestales con algunos cultivos; y, por último, terrazas de regadío con cultivos arbóreos en las cotas más bajas \autocite{JimenezOlivenciaetal2015MedioSiglo}. Además, otras actividades como la minería debieron alterar la estructura del bosque. Así por ejemplo, el entorno del robledal de San Juan (GEN) tiene muchas minas y canteras pequeñas que fueron explotadas intermitentemente a lo largo de la historia. De hecho, estos usos históricos se ven reflejados en los resultados de los análisis dendrocronológicos indicados en el capítulo \ref{sec:dendro}. Así por ejemplo, un evento de liberación del crecimiento arbóreo en la década de 1940 expresado en el registro dendrocronológico del robledal de San Juan, coincide con un periodo de máxima actividad minera en esta zona (1925 a 1957). Durante ese periodo se incrementó el uso de la madera para los túneles y hornos de las minas, que además requerían grandes cantidades de leña para fundir el mineral (\tabreft{tab:dendro:reviewusos}) \autocites{Titos1990}. Esta fuerte explotación de los recursos forestales en el entorno de estas minas, debió afectar a una parte importante de este robledal, como demuestra el crecimiento de los árboles remanentes en este robledal. Por otro lado, los bosques situados en la vertiente sur de Sierra Nevada (por ejemplo el robledal de Cáñar, CAN), se mezclaron con un mayor porcentaje de tierras de cultivo a lo largo del gradiente de elevación \autocite{JimenezOlivenciaetal2015MedioSiglo}. La leña, el carbón vegetal y las bellotas se explotaron intensamente en estos robledales, hasta al menos mediados del siglo XX, cuando estas actividades disminuyeron bruscamente debido principalmente al abandono rural y al uso de gas y combustibles fósiles \autocite{ValbuenaCarabanaGil2013GeneticResilience,MesaTorres2009,Bonet2014conama,MorenoLlorcaetal2016HistoricalAnalysis}. En las zonas altas de estos robledales, los registros dendrocronológicos indican una liberación de crecimiento en los primeros años registrados (en torno a 1830-1840) que podría estar relacionado con la conversión del bosque cerrado a un sistema silvopastoral abierto, una práctica de gestión común aplicada en el pasado en muchos robledales ibéricos \autocites{Canellasetal2004GrowthResponse,GeaIzquierdoetal2011TreeringsReflect} y que ha sido documentada para este sitio \autocites{ValbuenaCarabanaGil2013GeneticResilience}.

Sin embargo, a partir de la segunda mitad del siglo XX, se produjo un abandono de las actividades tradicionales \autocites{Piasetal2014ColonizationAbandoned,ValbuenaCarabanaetal2010HistoricalRecent,MartinezFernandezetal2015RecentLand,MacDonaldetal2000AgriculturalAbandonment}, que ha provocado una disminución de la presión antrópica sobre los ecosistemas forestales mediterráneos \autocites{ValbuenaCarabanaetal2010HistoricalRecent,PenuelasSardans2021GlobalChange}, siendo especialmente importante para las zonas de montaña \autocites{Nataleetal2007StudyTree, AlvarezMartinezetal2014InfluenceLand,JimenezOlivenciaetal2015MedioSiglo,Piasetal2014ColonizationAbandoned}. El dramático éxodo rural en las zonas de montaña se produjo debido a los cambios en las condiciones socioeconómicas \autocites{EuropeanEnvironmentAgency2010EuropeEcological}, dando lugar, además de al abandono de las actividades tradicionales, a importantes cambios ambientales \autocites{MacDonaldetal2000AgriculturalAbandonment, Nataleetal2007StudyTree, AlvarezMartinezetal2014InfluenceLand,Piussi2000ExpansionEuropean,Rutherfordetal2008AssessingLanduse,Zimmermannetal2010EffectsLanduse}. Así por ejemplo, varios estudios han demostrado que los cambios de usos están afectando ampliamente al almacenamiento de carbono de los ecosistemas terrestres en varias zonas del sur de Europa \autocite{MunozRojasetal2011ChangesLand,MunozRojasetal2015ImpactLand}. En este sentido, tras el abandono de las actividades tradicionales se ha observado una densificación de las masas forestales en áreas de montaña \autocite{JimenezOlivenciaetal2015MedioSiglo}. Los resultados del análisis de la evolución temporal de la biomasa arbórea en las parcelas del Segundo y Tercer Inventario Nacional Forestal (ver capítulo \ref{sec:carbon}), indican un aumento de la biomasa en el  89\% de las parcelas de \Qp en la Península Ibérica. Este resultado, junto con el incremento del área ocupada por esta formación en Sierra Nevada en las últimas décadas \autocite{CamachoOlmedoetal2002TransformacionPaisaje} y la constatación de la existencia de un proceso de densificación forestal \autocite{JimenezOlivenciaetal2015MedioSiglo}, ponen de manifiesto un incremento en la capacidad potencial de secuestro de carbono de estos bosques. Por tanto, considerando esta tendencia, y la casi ausencia de perturbaciones directas de origen humano, al estar estos bosques en una zona protegida, cabría esperar un aumento potencial de  las existencias de carbono forestal, como la que se ha registrado en muchos bosques de la región Mediterránea en las últimas décadas \autocite{FAOPlanBleu2018StateMediterranean}. Esto además coincide con las predicciones bajo diferentes escenarios que prevén un crecimiento forestal en las próximas décadas \autocite{Aparicioetal2015ClimateChange}. Sin embargo, debe tomarse con cautela, ya que se están documentando algunos signos de saturación de los bosques como sumideros de carbono \autocite{Nabuursetal2013FirstSigns}. 

Al igual que hemos apuntado para otras variables a lo largo de esta memoria doctoral, existe heterogeneidad en la biomasa aérea entre las poblaciones de roble en Sierra Nevada (ver capítulo \ref{sec:carbon}), condicionado fundamentalmente por las diferentes intensidades de uso a las cuales han estado sometidas dichas poblaciones en los últimos años. Así por ejemplo, la estructura forestal del robledal del Camarate (grupo N), con árboles más altos y más grandes, y por tanto mayores valores de biomasa,  podría relacionarse con una menor intensidad de las perturbaciones antropogénicas en comparación con los otros robledales. Los robledales del Camarate tuvieron una mayor protección durante la segunda mitad del siglo pasado \autocite{JimenezOlivencia1991PaisajesSierra}, y actualmente también tienen el mayor nivel de protección legal dentro del espacio protegido \autocite{Anonymous2011Decreto238}. Las menores perturbaciones antropogénicas han dado lugar a bosques mejor conservados con una mayor diversidad de especies \autocite{PerezLuqueetal2021EcologicalDiversity}, y también a una estructura de rodales estable con altos valores de biomasa (\figreft{fig:carbon:schema}). Para otras especies de \emph{Quercus}, también se ha observado que los bosques con menos perturbaciones tienen un mayor potencial de almacenamiento de carbono \autocite{BalboaMuriasetal2006CarbonNutrient,Cotillasetal2016AbovegroundBelowground,Stojanovicetal2017ForecastingTree}. Nuestros resultados ponen de manifiesto cómo las diferencias en la estructura del rodal condicionaron la biomasa arbórea del mismo. Los rodales muy densos (\emph{e.g.} poblaciones del NW) mostraron una biomasa total inferior a la de los rodales menos densos (poblaciones de roble del N o del S) (\figreft{fig:carbon:schema}). Una mayor densidad arbórea aumenta la competencia de los árboles, limitando el crecimiento de la misma, lo que provoca la pérdida de vitalidad y la reducción de la producción de bellota \autocite{Bravoetal2008SelviculturaMontes,Piqueetal2018Spain} y según nuestros resultados, cuanto mayor es la densidad de los árboles, menor es la capacidad de estos bosques para actuar como sumideros de carbono (Figura \ref{fig:carbon:glm}). Además, esta mayor acumulación de biomasa, unida a una pérdida de diversidad estructural, aumenta el riesgo de incendios forestales debido a la gran cantidad de biomasa acumulada \autocite{Canellasetal2004GrowthResponse,PiqueVericat2015EvolutionPerspectives,Serradaetal1992CoppiceSystem}.

El abandono de las actividades tradicionales en montaña, como por ejemplo los cultivos de montaña, también está provocando un aumento de la expansión forestal hacia las tierras de cultivo abandonadas \autocites{Piussi2000ExpansionEuropean, AlvarezMartinezetal2014InfluenceLand}, que puede provocar una homogeneización del paisaje \autocites{Mietkiewiczetal2017LongtermChange} con diversas consecuencias ecológicas \autocites{Zimmermannetal2010EffectsLanduse}. Así, a pesar de las fuertes limitaciones de reclutamiento descritas para esta especie \autocites{Bravoetal2008SelviculturaMontes,Gomez2003ImpactVertebrate,Pereaetal2014InteraccionesPlantaanimal}, hemos constatado la existencia de un proceso de colonización de \Qpy hacia tierras de cultivo abandonadas en Sierra Nevada (ver capítulo \ref{sec:coloniza}). Este fenómeno de expansión forestal hacia las tierras de cultivo abandonadas también se ha registrado en otras regiones montañosas europeas \autocites{Amezteguietal2016LanduseLegacies,Nataleetal2007StudyTree,Piussi2000ExpansionEuropean,Amezteguietal2010LanduseChanges,LasantaMartinezetal2005MountainMediterranean,Kozak2003ForestCover,AlvarezMartinezetal2014InfluenceLand,VicenteSerranoetal2004AnalysisSpatial} como consecuencia principalmente de la despoblación rural y de la disminución de la presión de los herbívoros \autocite{MacDonaldetal2000AgriculturalAbandonment,EuropeanEnvironmentAgency2016EuropeanForest}. En Sierra Nevada, hemos observado que el proceso de colonización de cultivos por parte del robledal es diferente entre las poblaciones estudiadas. Encontramos que en las poblaciones del sur (Robledal de Cáñar), la cantidad de juveniles de \Qp en el interior de los cultivos abandonados era mayor que en las poblaciones de la vertiente norte. De los diferentes aspectos analizados que puedan explicar esas diferencias, encontramos que la distancia a la fuente semillera, la estructura del bosque circundante a los cultivos abandonados, y la población de arrendajo (\emph{Garrulus glandarius}, el principal dispersante de bellotas del roble) no presentan diferencias significativas para las poblaciones estudiadas (ver capítulo \ref{sec:coloniza}). 

Además de la importancia de los factores relacionados con la dispersión y con la variación a escala fina de los factores abióticos \autocite{Milderetal2013ColonizationPatterns,Leverkusetal2016ShiftingDemographic}, 
la historia de uso a la que han sido sometidos los cultivos de montaña abandonados (antes y después de su abandono) es un factor clave que determina la abundancia de las especies arbóreas nativas \autocites{HermyVerheyen2007LegaciesPresentday,NavarroGonzalezetal2013WeightLanduse,AlvarezMartinezetal2014InfluenceLand,Perringetal2016GlobalEnvironmental}. La historia de uso y gestión de nuestros sitios de estudio, inferida a partir de varios trabajos de recopilación histórica \autocites{MorenoLlorcaetal2014CaracterizacionFuentes, Titos1990, PerezLuqueetal2020LanduseLegacies,MorenoLlorcaetal2016HistoricalAnalysis,MesaTorres2009,JimenezOlivenciaetal2015EvolucionUsos}, señala que ambos sitios estuvieron sometidos a intensos usos antrópicos en el pasado. En las poblaciones de la vertiente norte (por ejemplo, robledal de San Juan), las zonas altas estaban dedicadas al pastoreo, y en las áreas de bosque también había algunas tierras de cultivo con pastoreo, mientras que el robledal de Cáñar (vertiente sur) ha sido explotado para leña, carbón vegetal y bellotas, con menor presencia de uso ganadero. Aunque no hemos podido estimar la intensidad de uso a la que han estado sometidas ambas zonas antes del abandono de los cultivos, la zona norte parece haber tenido una historia de manejo con mayor intensidad de pastoreo que el sitio sur \autocite{MorenoLlorcaetal2016HistoricalAnalysis, MorenoLlorcaetal2014CaracterizacionFuentes,MorenoLlorcaZamora2012CaracterizacionCarga}. Por otro lado, es importante conocer la historia de uso tras el abandono del cultivo de montaña, centrándonos en la presión ganadera, ya que la herbivoría impone severas limitaciones al establecimiento y regeneración de esta especie \autocites{Gomez2003ImpactVertebrate, Pereaetal2014InteraccionesPlantaanimal}. Aunque no se dispone de datos sobre la evolución temporal de la presión de pastoreo a escala de detalle en nuestros lugares de estudio, varios estudios e informes, combinando entrevistas con pastores y revisión de documentos históricos, han inferido la historia ganadera reciente en varios robledales de Sierra Nevada \autocites{MorenoLlorcaetal2016HistoricalAnalysis, MorenoLlorcaetal2014CaracterizacionFuentes,MorenoLlorcaZamora2012CaracterizacionCarga}. Así, se ha observado tanto un mayor número de rebaños y pastores, como una mayor densidad ganadera en el robledal de San Juan (vertiente norte) que en el de Cáñar (vertiente sur), lo que podría traducirse en una mayor presión herbívora. Esta distinta presión herbívora podría explicar las diferencias observadas en la abundancia de robles juveniles tanto en el interior de los cultivos abandonados como en el borde y en el interior del bosque circundante (ver  \figreft{fig:coloniza:treeCategory}). La herbivoría, más que los factores abióticos, es la principal causa de mortalidad de plántulas de \Qp en Sierra Nevada \autocite{Gomez2003ImpactVertebrate}, pero las plántulas mueren en su mayoría por el efecto del pisoteo por parte del ganado silvestre y doméstico, más que por el ramoneo (que es una causa de muerte marginal para las plántulas de roble) \autocite{Gomez2003ImpactVertebrate}. 

Las diferencias en los patrones de recolonización dentro del borde posterior parecen estar relacionadas con las diferencias en la gestión antes y después del abandono de las tierras de cultivo de montaña. Una mayor presión de herbivoría tras el abandono de las tierras de cultivo parece limitar la expansión del bosque hacia los hábitats marginales. En este sentido, y con el fin de mejorar la expansión del bosque, sería recomendable aprovechar la presencia de arbustos nativos que ofrecen lugares seguros ayudando a reducir la mortalidad de las plántulas de \Qp, y por lo tanto aumentar las probabilidades de establecimiento de esta especie. Esto también ayudaría a aumentar la heterogeneidad en el desarrollo del bosque secundario que se está estableciendo, lo que aumentaría la resiliencia a las perturbaciones y la recuperación de la multifuncionalidad del ecosistema \autocite{Stritihetal2021ImpactLanduse, CruzAlonsoetal2019LongTerm}. Por otro lado, también es necesario prestar atención al mantenimiento de las fuentes semilleras en buen estado de salud (bosques de alrededor), y de una comunidad estable de dispersores de semillas, particularmente del arrendajo, ya que la dispersión de bellotas por parte de esta especie de ave se considera un proceso clave en la regeneración de los bosques de \emph{Quercus} tras el abandono del terreno \autocite{Pausasetal2006RegenerationMarginal}.


\subsection*{Vulnerabilidad de las poblaciones que viven en los márgenes de distribución: el caso de los robledales de Sierra Nevada
}\label{sec:discussions:rear}

Las poblaciones de \Qp situadas en su borde posterior de distribución presentan una alta sensibilidad a la disponibilidad de agua, siendo generalmente el factor que más limita el crecimiento secundario de esta especie \autocite[\emph{e.g.} capítulo \ref{sec:dendro}, ][]{GeaIzquierdoCanellas2014LocalClimate}. Esta sensibilidad a las variables relacionadas con la humedad se ha observado también en otras especies arbóreas cuyas poblaciones se sitúan en su margen posterior de distribución \autocite[\emph{e.g. Abies alba}][]{MartinezSanchoGutierrezMerino2019EvidenceThat}. Sin embargo, otras especies son más sensibles a las temperaturas \autocite[\emph{e.g.} \emph{Pinus sylvestris},][]{Herreroetal2013VaryingClimate} o responden simultáneamente a las variables relacionadas con la temperatura y la humedad \autocites[\emph{e.g.} \emph{Fagus sylvatica},][]{DoradoLinanetal2017CoexistenceMediterraneanTemperate,DoradoLinanetal2017ClimateThreats}[\emph{Pinus nigra} subsp. \emph{salzmanii},][]{SanchezSalgueroetal2012DroughtMain}. Esta diversidad en la respuesta de las especies arbóreas a la precipitación y a la temperatura, sugiere que la vulnerabilidad al cambio climático no se expresa de forma consistente dentro del borde posterior, evidenciando por tanto que los bosques geográficamente marginales no son necesariamente climática o ecológicamente marginales \autocite[ver][ y las referencias en dicho trabajo]{DoradoLinanetal2019GeographicalAdaptation}.

En el capítulo \ref{sec:dendro}, combinado el uso de la teledetección, la dendrocronología, y la revisión de documentos históricos, hemos constatado que las poblaciones de roble en Sierra Nevada presentan unos altos valores de resiliencia a la sequía (crecimiento primario y secundario). Estos valores junto con el papel potencial de la adaptación local \autocite[\emph{e.g.} altos valores de resiliencia genética para los robledales de Sierra Nevada, ][]{ValbuenaCarabanaGil2013GeneticResilience}, sugieren que la historia de gestión (usos del suelo) también tiene un papel clave para determinar la resiliencia de los árboles a la sequía en el borde posterior de distribución. Nuestros resultados coinciden con otros estudios que muestran que la supuesta mayor vulnerabilidad a la sequía en las poblaciones situadas en los márgenes geográficos de distribución actual no necesariamente se mantienen  \autocite[\emph{e.g.}][]{CavinJump2017HighestDrought}. En nuestro caso, esto puede explicarse por el hecho de que el actual borde posterior de distribución geográfica no coincide con el potencial borde posterior ecológico de la especie, ya que éste ha sido modificado y determinado en su mayor parte por la intervención del hombre (\emph{i.e.} historia de uso). 

Los valores de resiliencia, resistencia y recuperación de los robledales de Sierra Nevada tras los eventos de sequía están fuertemente influenciados por la orientación y las condiciones ambientales locales, así como por la historia de manejo de los mismos. La adaptación y plasticidad local de algunas especies, así como la variación de los factores ambientales a escala local, se consideran factores importantes que determinan la vulnerabilidad de algunas especies en los bordes posteriores de distribución \autocites{MartinezVilalta2018RearWindow}. Asimismo, hemos encontrado una amplia variabilidad de los valores de resiliencia a pequeña escala en los robledales en Sierra Nevada. Estos resultados sugieren que los márgenes posteriores de distribución geográficos y ecológicos no necesariamente coinciden, y que a escalas espaciales más pequeñas, la vulnerabilidad frente al cambio climático puede variar dentro del propio margen posterior de distribución de la especie. 

Las poblaciones de \Qp en Sierra Nevada están localizadas en un borde marginal geográfico, pero no en un borde marginal ecológico \autocites[\emph{sensu}][]{MartinezVilalta2018RearWindow,VilaCabreraetal2019RefiningPredictions}. Contrariamente a lo esperado, los robles mostraron una alta resiliencia en respuesta a la sequía, especialmente a largo plazo. Estos altos valores pueden estar relacionados con mecanismos estabilizadores que promueven la resiliencia de los individuos adultos ya establecidos \autocite[\emph{e.g.} capacidad de tolerancia al estrés vinculada a la adaptación local][]{Lloretetal2012ExtremeClimatic}, y que pueden estar amortiguando el impacto de los eventos climáticos extremos, como se ha descrito para otras especies \autocite[\emph{e.}{Pinus sylvestris},][]{HerreroZamora2014PlantResponses}. 

Las respuestas de resiliencia del robledal a los eventos de sequía no son espacialmente homogéneas en Sierra Nevada, debido a las diferencias en las condiciones ecológicas y/o a los legados de gestión del pasado. De hecho, existe mucha variabilidad a pequeña escala en la respuesta al clima a lo largo del borde posterior de distribución que no habíamos considerado \emph{a priori}. Las diferencias encontradas en el crecimiento de los árboles, la sensibilidad climática y la resiliencia de los árboles entre sitios muy cercanos mostraron que las respuestas a la sequía dependen del sitio, y pueden variar drásticamente en gradientes espaciales extremadamente estrechos. En las regiones montañosas, la heterogeneidad de las condiciones ecológicas a escalas finas es la regla, lo que permite la existencia de microrrefugios y la prolongación de la persistencia de las especies \autocite{Olaldeetal2002WhiteOaks,SerraDiazetal2015DisturbanceClimate}. Esto es especialmente relevante para definir la extensión real y la naturaleza (geográfica y/o ecológica) de las poblaciones del borde posterior, donde la variabilidad topográfica y biofísica facilita la existencia de microrrefugios.

Por otro lado, el análisis de la dinámica de crecimiento arbóreo reveló eventos de supresión y liberación que eran consistentes con los legados de uso inferidos de la revisión exhaustiva de documentos históricos. Esto sugiere que el concepto de borde posterior necesita ser redefinido en el espacio, pero también en el tiempo \autocite{VilaCabreraetal2019RefiningPredictions}, en parte debido a la importancia de los los legados de uso y su efecto en el posible desajuste entre la distribución actual de las especies (\emph{i.e.} determinando el borde posterior ``geográfico disponible'') y el borde posterior ecológico potencial (limitante) de las especies. El concepto de retaguardia o de borde posterior de distribución (\emph{i.e.} \emph{rear edge}) también debería considerar aspectos históricos además de los geográficos, climáticos y genéticos \autocite{VilaCabreraetal2019RefiningPredictions}, especialmente en áreas con una larga historia de gestión humana, como las montañas Mediterráneas. Por lo tanto, la modificación antropogénica del hábitat y sus legados representan una dimensión crítica de la marginalidad, ya que pueden intensificar, confundir o retrasar el declive poblacional impulsado por el clima en los bordes posteriores de distribución \autocite{VilaCabreraetal2019RefiningPredictions}. Esto es relevante para las especies arbóreas muy sensibles al cambio climático, como \Qpy, no sólo para la conservación \emph{per se} de la especie, sino para todos los servicios ecosistémicos que ofrecen estos bosques. En este sentido, sería necesario analizar la resiliencia de todos los estadios demográficos de las especies, para asegurar que la resiliencia observada en los árboles adultos se manifiesta también en su dinámica de reclutamiento demográfico expresada por la regeneración natural. La resiliencia también podría ser diferente para las distintas cohortes de edad o en las plántulas en comparación con los rebrotes.

\subsection*{Provisión de servicios ecosistémicos de los robledales de Sierra Nevada
}\label{sec:discussions:es}
Como se ha apuntado a lo largo de esta memoria doctoral, el conocimiento de la dinámica de funcionamiento de los robledales de Sierra Nevada, es fundamental para desarrollar estrategias de gestión adecuadas bajo las incertidumbres climáticas actuales \autocites{Fadyetal2016EvolutionbasedApproach,Jumpetal2010MonitoringManaging}. Las poblaciones situadas en los bordes posteriores de su distribución, merecen \emph{per se} una atención especial debido a su alto valor de conservación \autocite{Fadyetal2016EvolutionbasedApproach}. Son poblaciones que suelen estar adaptadas a las condiciones ambientales locales en el límite de la amplitud ecológica de la especie, y a menudo muestran una persistencia a largo plazo \autocite{HampePetit2005ConservingBiodiversity}. Además, las respuestas locales a los cambios ambientales pueden diferir de la respuesta media de la especie \autocites{Benavidesetal2013DirectIndirect,Matiasetal2017ContrastingGrowth,Castroetal2004SeedlingEstablishment}, y estas diferencias pueden favorecer o dificultar la supervivencia de las poblaciones situadas en los límites de distribución en los escenarios de cambio global \autocites{Fadyetal2016EvolutionbasedApproach,Jumpetal2010MonitoringManaging}. 

Un aspecto importante en el estudio de las poblaciones situadas en los límites de distribución, además de su dinámica de funcionamiento, es la provisión de servicios ecosistémicos considerando tanto los usos del suelo como los escenarios de cambio climático. Como hemos visto en los capítulos \ref{sec:carbon} y \ref{sec:es}, además es importante tener en cuenta cómo ha variado a lo largo de las últimas décadas esta provisión de servicios ecosistémicos; y de cara a la gestión actual, la variabilidad existente de provisión de servicios ecosistémicos entre las diferentes poblaciones de robledal, que pueden ayudar a dirigir las actuaciones de gestión de estas formaciones forestales. 

Con respecto al papel de estos bosques como potencial sumidero de carbono, en el capítulo \ref{sec:carbon} hemos estimado que los robledales de Sierra Nevada presentan unos altos valores de biomasa forestal aérea (104.69 - 111.71 \mgha, ver \tabreft{tab:carbon:compara}). Estos resultados son coherentes con los estimados para los ecosistemas forestales montañosos por el IPCC para Europa \autocite[130 (20 - 600) \mgha; ][]{IPCC2006ForestLand}. Sin embargo, nuestros resultados discrepan con las estimaciones realizadas por diferentes autores para robledales a lo largo de su rango de distribución. Por ejemplo, \citet{Vayredaetal2012SpatialPatterns}, usando datos del Tercer Inventario Forestal Nacional de España, encontró para los robledales de \Qp, un valor medio para el stock de carbono de 45 \mgha. Estos valores eran inferiores al stock de carbono estimado en nuestro estudio, que oscilaba entre 69.95 y 74.63 \mgha. A una escala más regional, nuestros resultados mostraron valores más altos que los encontrados por otros trabajos realizados en el rango central de la distribución de la especie, donde la biomasa aérea varió entre 63,8 - 98 \mgha \autocite{GallardoLanchoGonzalezHernandez2004SequestrationCarbon}. Asimismo, otras estimaciones del secuestro de dióxido de carbono en rodales puros de \Qp situados en el Sistema Central (Península Ibérica), arrojaron valores inferiores a los nuestros. A pesar de las posibles diferencias derivadas del método de estimación del carbono (\emph{e.g.} estimación utilizando LIDAR \emph{vs.} estimación usando medidas en campo), otros factores podrían explicar las diferencias encontradas en nuestros resultados con respecto a los valores reportados para otros estudios. En primer lugar, está generalmente aceptado que existe un declive relacionado con la edad en la acumulación de biomasa de los rodales \autocite[][y referencias incluidas en ese trabajo]{Xuetal2012AgerelatedDecline}, siendo la productividad de los bosques viejos generalmente menor que la de los bosques más jóvenes \autocite{Kutschetal2009EcophysiologicalCharacteristics}. Los robledales de Sierra Nevada están compuestos por árboles relativamente jóvenes \autocite{GeaIzquierdoCanellas2014LocalClimate,PerezLuqueetal2020LanduseLegacies,RubioCuadradoetal2018AbioticFactors} en comparación con otros bosques de la especie a lo largo de su área de distribución \autocite{GeaIzquierdoCanellas2014LocalClimate}. Como hemos visto en esta memoria doctoral, las fuertes perturbaciones antrópicas en estos robledales han condicionado su estructura. Por ejemplo, algunos de los robledales fueron masivamente talados durante la época de la posguerra para su uso como gasógeno para los vehículos \autocite[\emph{e.g.} robledal San Jerónimo, MON;][]{Prieto1975BosquesSierra}, o para su uso en intensas actividades mineras \autocite[\emph{e.g.} fundición de mineral en el entorno del robledal de San Juan, GEN][]{PerezLuqueetal2020LanduseLegacies}. Por lo tanto, podemos considerar que muchos de los robledales de Sierra Nevada son relativamente jóvenes, lo que podría explicar el alto potencial de acumulación de C obtenido en nuestro estudio, ya que se ha demostrado que los bosques creados como resultado de cambios drásticos en el uso del suelo exhiben tasas de crecimiento más rápidas, y por lo tanto una mayor acumulación potencial de C, que los bosques preexistentes \autocite{VilaCabreraetal2017NewForests}. Asimismo, otros estudios han mostrado diferencias en el stock de carbono entre formaciones jóvenes y bosques maduros para diferentes especies de \emph{Quercus} \autocite{Bruckmanetal2011CarbonPools,Cotillasetal2016AbovegroundBelowground}. Por tanto, es probable que los altos valores de C estimados para los robledales de Sierra Nevada pueda explicarse en parte por el estado de desarrollo de estas masas forestales \autocite{Makinecietal2015EcosystemCarbon}. En segundo lugar, la disponibilidad de agua es generalmente el factor más limitante que impulsa el crecimiento radial del \Qp a lo largo de su rango de distribución en la Península Ibérica \autocite{GeaIzquierdoCanellas2014LocalClimate}. En Sierra Nevada, los robledales de la vertiente N y NW se localizan en fondos de valle con altos valores de humedad relativa, mientras que las de la vertiente S reciben el aporte extra de agua del aire húmedo procedente del mar Mediterráneo. Por lo tanto, la disponibilidad de agua no parece estar limitando el crecimiento del roble en esta región montañosa. De hecho, como se comentó previamente, se han observado tendencias positivas para el verdor (EVI) y el crecimiento secundario de los robledales de Sierra Nevada, sugiriendo que Sierra Nevada podría actuar como un refugio ecológico para esta especie. Estas tendencias positivas de crecimiento podrían explicar los altos valores de secuestro de carbono obtenidos en para las poblaciones de robledal de Sierra Nevada en comparación con las aportadas por otros estudios, aunque sería necesario realizar un estudio más detallado de las tendencias de crecimiento y la biomasa comparando entre poblaciones situadas en el centro de su distribución y aquellas localizadas en los bordes de distribución.  

En el capítulo \ref{sec:es} hemos llevado a cabo una revisión exhaustiva de los principales servicios ecosistémicos que proporcionan los bosques de roble melojo a nivel general. La recopilación bibliográfica realizada ha mostrado que los servicios ecosistémicos que han sido evaluados con mayor frecuencia corresponden con los servicios de aprovisionamiento. Destacan los estudios centrados en investigar el efecto de la madera de melojo en el proceso de elaboración del vino \autocites[\emph{e.g.}][]{FernandezdeSimonetal2010CharacterizationVolatile,CastroVazquezetal2013EvaluationPortuguese}, ya que las barricas se construyen frecuentemente con la madera de esta especie. Además, encontramos diferentes estudios en los que se evalúa la provisión de setas \autocites[\emph{e.g.}][]{OriadeRuedaetal2010CouldArtificial}, o la producción de madera o biomasa para energía  \autocites[\emph{e.g.}][]{Mirandaetal2009EnergeticCharacterization}. En cuanto a los servicios de regulación, varios estudios evalúan el papel de los robledales en la calidad y fertilidad del suelo, o su capacidad de secuestro y almacenamiento de carbono \autocites[\emph{e.g.}][]{Alvarezetal2014InfluenceTree}. También hay una alta proporción de estudios sobre el carbono del suelo \autocites[\emph{e.g.}][]{Fonsecaetal2019ImpactTree}. Por último, cabe destacar que con los criterios de búsqueda utilizados (ver capítulo \ref{sec:es}), no se ha encontrado ningún estudio sobre la evaluación de los servicios culturales en los bosques de \Qpy.

Posteriormente, utilizando las poblaciones de roble melojo de Sierra Nevada como caso de estudio, hemos explorado la variación espacio-temporal de la provisión de servicios ecosistémicos por parte de esta formación forestal. Combinamos conocimiento experto y diferentes fuentes de datos (literatura gris, datos de proyectos de investigación,  programas de seguimiento, etc) para cuantificar en la medida de lo posible los servicios ecosistémicos proporcionados por estos bosques. Respecto al patrón espacial, hemos podido observar cómo existen diferencias de provisión de servicios ecosistémicos entre las poblaciones de robledal en Sierra Nevada. Los robledales de la vertiente sur de Sierra Nevada presentan mayores valores de servicios de regulación, mientras que los robledales situados en la vertiente norte exhiben mayores valores para los servicios culturales. Nuestra recopilación de datos a nivel local nos ha permitido cuantificar muchos de los servicios ecosistémicos suministrados por los bosques de \Qp en Sierra Nevada, lo que podría ayudar a los gestores de recursos naturales con más información y herramientas para ayudarles en el proceso de toma de decisiones.

En cuanto a los servicios de regulación evaluados, observamos que los robledales situados en la vertiente sur de Sierra Nevada presentan valores más altos para el potencial de secuestro de carbono, el EVI medio y el carbono orgánico del suelo que el resto de poblaciones de robledal, a pesar de la esperada mayor vulnerabilidad debido a su localización en las zonas más meridionales de Sierra Nevada. Estos resultados están en consonancia con los resultados encontrados en otros trabajos (ver capítulo \ref{sec:dendro}), que destacaban mayores valores de resiliencia a las perturbaciones de los robledales de la vertiente sur. Para los servicios culturales, los robledales situados en el noroeste de Sierra Nevada presentan valores elevados para las actividades deportivas, el número de visitantes y la densidad de los protocolos de muestreo (valor científico). Finalmente, para los servicios de provisión, observamos un patrón variable en función del servicio evaluado. Esta cuantificación de algunos de los servicios ecosistémicos proporcionados por las poblaciones de roble melojo en Sierra Nevada, nos ha permitido comparar el suministro de servicios ecosistémicos entre dichas poblaciones, y describir qué categoría de servicios ecosistémicos predomina en cada uno de los grupos de poblaciones de roble en Sierra Nevada. Por otro lado, hemos observado un patrón temporal de suministro de servicios ecosistémicos condicionado principalmente por los usos antrópicos a los que ha estado sometido esta formación forestal. Hasta la década de 1970 el suministro de servicios ecosistémicos que predominaba en esta formación era el de abastecimiento y provisión. Aunque los servicios de regulación podían estar presentes (\emph{e.g.} control de la erosión en algunas zonas escarpadas, por el característico sistema radicular de esta especie), los robledales eran usados como proveedores de leñas, carboneo, pastos para el ganado, bellotas, entre otros (\emph{i.e.} servicios de provisión y abastecimiento). Poco a poco se fueron abandonando algunas actividades tradicionales, lo que provocaba la disminución del suministro de algunos servicios de abastecimiento frente a un ligero incremento de los servicios de regulación por parte del bosque. Posteriormente tras el abandono generalizado de las actividades tradicionales, se produjo una reducción drástica en la provisión de servicios de abastecimiento. Asimismo, debido en parte a la expansión del bosque hacia cultivos abandonados, y a la densificación de las masas de robledal existente, se aumentó la provisión de servicios de regulación (\emph{e.g.} regulación climática, secuestro de carbono). Por otro lado, fruto de las políticas de protección de los recursos naturales (\emph{e.g.} declaración de espacio natural protegido) y de un aumento de la concienciación sobre los valores ambientales que proporciona la naturaleza \autocite{Mace2014WhoseConservation}, entre otras razones, se produjo un aumento de la provisión de servicios culturales por parte de estas formaciones forestales. 

Además de la evolución temporal de la provisión de servicios ecosistémicos, nuestro trabajo también añade algunas ideas interesantes para el estudio de los servicios ecosistémicos proporcionados por los robledales. Por ejemplo, observamos un patrón temporal muy variable de algunos servicios ecosistémicos, como el valor recreativo. Algunos bosques de melojo de Sierra Nevada concentran gran parte de los visitantes registrados en el Espacio Natural de Sierra Nevada durante un periodo de tiempo concreto. Esto pone de manifiesto la necesidad de considerar la dimensión temporal de los servicios ecosistémicos evaluados, y de prestar atención a la presión que estos ecosistemas pueden estar sufriendo temporalmente debido al elevado número de visitantes. Por lo tanto, desde el punto de vista de la gestión de los recursos naturales, no sólo es necesario analizar los servicios ecosistémicos que proporciona un ecosistema, sino que también habría que considerar el patrón espacio-temporal de la oferta de dichos servicios. En este sentido, sería interesante realizar estudios detallados para proporcionar a los gestores una evaluación completa de los posibles impactos de los visitantes. Consideramos por tanto, que nuestra recopilación de datos a nivel local nos ha permitido cuantificar muchos de los servicios ecosistémicos suministrados por los bosques de \Qp, lo que proporciona a los gestores información y herramientas clave para ayudarles en el proceso de toma de decisiones.