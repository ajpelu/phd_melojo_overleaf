% !TEX root = ../my-thesis.tex
%
\pdfbookmark[0]{Acknowledgement}{Acknowledgement}
\addchap*{Agradecimientos}
\label{sec:acknowledgement}

\small
En primer lugar quiero agradecer a mi familia por sostenerme siempre. Mamá, Papá (\dag), os agradezco infinitamente todo el esfuerzo que habéis realizado para brindarnos a mis hermanos y a mí, a pesar de las dificultades inherentes a una familia humilde, el acceso a una educación reglada, toda la que hemos soñado. Más importante ha sido para mí la educación en valores que he recibido de vosotros: el sacrificio, la responsabilidad, la honestidad, la justicia y la generosidad, han sido algunas de las virtudes que me habéis inculcado con vuestro ejemplo. Espero haber sido un buen alumno y haber incorporado a mi vida algo de lo que he visto en vosotros. Gracias Mamá por siempre saber salir hacia delante. Papá me encantaría que pudieras leer algo de esta tesis. Te echo de menos. 

Gracias a mi esposa, Mari Ángeles, y a mis hijos Elena María y David. Vosotros habéis sufrido mis ausencias durante este camino. Gracias Mari Ángeles, sin ti la tesis no hubiera sido posible. Gracias por caminar conmigo a mi lado, cogiéndome la mano en todo momento. Gracias por tirar hacia delante siempre, con una sonrisa. Bien sabemos que mucho de esta tesis se debe a tí. Gracias por llevar el peso de la familia durante tanto tiempo, has soportado mis momentos más críticos y me has ayudado a salir a flote. Gracias por amarme. Doy gracias a Dios, por habernos puesto a caminar juntos. David, Elena, gracias por ser como sois y por entender que papá ha estado ocupado durante demasiado tiempo. Os quiero mucho. Elena, gracias por tu dibujo, es una preciosidad, para mí significa mucho que hayas querido ilustrar este trabajo.

Gracias también a mis hermanos Encarni, Rafael, y Sandra. Gracias por no agobiarme demasiado con la típica pregunta \emph{¿cuándo vas a acabar la tesis?}. Cuñao!! gracias por compartir momentos en la montaña. A mis sobrinos Antonio, Abraham, Sofia, Irene, Rafael y Manuela. Os echo de menos mucho. 

Mi más sincero agradecimiento para mis directores de tesis Regino Zamora y Francisco J. Bonet. Gracias por brindarme la oportunidad de realizar esta tesis doctoral, y por haber facilitado y sostenido el trabajo que hemos desarrollado en el transcurso de la tesis. Destaco la confianza que depositasteis en mí, y la libertad que me habéis proporcionado para desarrollar este trabajo. Esa libertad, aunque a veces me ha generado incomodidad interior, me ha servido para desarrollar más en profundidad mis inquietudes científicas sobre la ecología de los robledales. Ha sido un placer para mí poder desarrollar este trabajo de la mano de un gran ecólogo como es Regino Zamora. Gracias Regino por tus aportaciones a los diferentes capítulos, siempre tan acertadas. Curro, de ti aprendí muchas mas cosas personales que ecológicas,  ….. 

Durante el proceso de elaboración de esta tesis doctoral, he tenido la oportunidad de compartir buenos (muy buenos) y otros menos buenos momentos con mis compañeros de iEcoLab. Son muchas las personas que han pasado por allí y con las cuales he tenido el privilegio de compartir muchas horas de trabajo. Nacho Villegas, Curro Cabezas, Irene, Lucía, Susana Hitos, Ramón, Raúl Casado, Fabio, Luis Cayuela, Pablo González, Ricardo, Pepa, Pablo S. Reyes, Blas M. Benito, Fran, Dani, María Suárez, Andrea, Manuel Merino, y Pablo. Gracias Pepa por el día a día, gracias por ayudarme haciéndome la vida más fácil con el papeleo. Gracias por escucharme en los momentos más caóticos, y por capturar tan espléndidamente la belleza de la natura con tu cámara, reflejo de una delicada sensibilidad, que pocos poseen. Gracias de corazón. Ramón te debo mucho. De tí aprendí gran parte de lo que hoy conozco relacionado con la ecoinformática, y más importante, tu cercana presencia me ayudó durante mucho tiempo. Gracias por tu disponibilidad absoluta. Amigo Blas, gracias por iniciarme en los primeros momentos de la investigación. ... 


Gracias también a los que habéis pasado algún tiempo allí con nosotros: Sammy, Eladia, Lola Álvarez, Carlos A. Saavedra, Maite, Grettel, Wendy, Aberto. De todos vosotros he aprendido algo. Gracias a todos. No me olvido de los “vecinos” dentro del CEAMA, en especial Javi Herrero, Pedro y Tino. He tenido la oportunidad de colaborar con vosotros (menos de lo deseado), y agradezco vuestra amabilidad y disposición. 




María Suárez y Sammy me ayudaron en algunos de los muestreos de campo. Gracias María por tu punto de vista crítico y perspicaz. Ha sido inspirador aunque no te lo haya dicho (aunque ya queda por escrito). Sammy, la mezcla de rigidez alemana con el punto de locura de tu personalidad fueron un cóctel muy refrescante en los momentos de campo. Gracias a todo el personal que sostiene y facilita las tareas de investigación. Gracias a Antonio, Maria Jesús y Lola, nuestros maravillosos conserjes del CEAMA, que todos los días nos regalaban una sonrisa para comenzar el día. Gracias al equipo de administración: Luis y Mariam, y a los directores Miguel Losada y Lucas Alados. Gracias también al equipo de limpieza que hacían que nuestro espacio de trabajo estuviera en condiciones óptimas para el desempeño de nuestra labor. Y no me puedo olvidar de los compañeros de “La Cabaña”, en especial de Juan (la persona que más sabe del CEAMA) y Víctor (nunca seré capaz de arrebatarle la última palabra). Gracias por vuestros consejos. 

No quiero olvidarme de algunos integrantes del Departamento de Ecología de la Universidad de Granada, con los que he compartido algunos interesantes momentos: Ana Mellado, Alba (gracias por los consejos, y por la ayuda con la plantilla de \LaTeX{}), Alex Leverkus. Gracias a José Antonio Hódar por sus perpetuos ánimos. Gracias también a los profesores Carmen Pérez, Isabel Reche, Mani, Pre, José M. Gómez, José M. Conde, Jorge, Rafa Morales, y antiguos miembros: Carolina, Oti, Eva Ortega, Mati, Irene, Jose Luis, Lorena, Asier, ... De alguna forma habéis sido inspiradores y habéis alimentado mi interés por diversos aspectos de la ecología. Gracias. 

Estoy especialmente agradecido al tutor del programa de doctorado, Juan Manuel Medina. Gracias JuanMa por tu disponibilidad, cercanía y tu buen hacer, así como tus comentarios para un mejor aprovechamiento del programa de doctorado. Agradezco también los sabios comentarios sobre diversos aspectos del doctorado de Jose Carlos Prados (Catedrático del Departamento de Anatomía y Embriología). Jose, gracias por tus consejos. 

Estoy inmensamente agradecido al personal que forma o ha formado parte del Espacio Natural de Sierra Nevada, que me han facilitado el trabajo en nuestra maravillosa Sierra. Gracias por cuidar y gestionar tan bien este tan preciado espacio natural. Mi agradecimiento especial a Javier Sánchez, Ignacio Henares, Blaca Ramos y Francisco J. Cano. Javier Navarro y Javier Cano me proporcionaron información muy valiosa sobre proyectos de gestión históricos de robledales. El equipo del área de Uso Público y del área de Agentes forestales, que proporcionaron amablemente información sobre los sensores piroeléctricos (en especial gracias a José Manuel Castilla, José Pino y a José Pedro Sánchez Martínez). De igual modo, a todo el equipo de Agentes Forestales, en especial a Juan Reyes. Gracias por vuestra disponibilidad eterna. A los compañeros de la Agencia Andaluza de Medio Ambiente y Agua con los que he compartido muchos momentos en el Observatorio de Cambio Global de Sierra Nevada. Gracias a: José Miguel Muñoz, Miguel Arrufat, Rogelio, Cristina P. Sánchez, Mónica, Mariqui, Miguel Galiana, Nano, Ernesto Sofós, Juande, Gonzalo. Agradezco especialmente a Antonio Veredas su ayuda en algunos muestreos. Además de conocer enormemente la Sierra, trabajar con él es enormemente satisfactorio por su afable personalidad y lo fácil que lo hace todo. Gracias Antonio. He aprendido mucho de tí. Especial mención merecen Rut Aspizua, José Miguel Barea y Javier Cano con los que he tenido la inmensa fortuna de aprender juntos durante la etapa del proyecto LIFE-ADAPTAMED.  
... 

Gracias al grupo SERPAM, de la EEZ-CSIC, por acogerme en esta última etapa de la tesis. Gracias Ana, Mariu, Mauro y José Luis, por entender mi situación en el momento final de la tesis. Os agradezco mucho la forma en la que me habéis acogido. Me sobrevalorais en demasía. 

Gracias a los que han colaborado más estrechamente en algunos capítulos de esta tesis doctoral. En especial quiero agradecer además de a mis directores, a Guillermo Gea-Izquierdo, a José V. Roces-Diez ("Pipo"), a Mihai Tanase y a Cristina Aponte, a Ramón Pérez, Blas M. Benito y Pedro J. Magaña. Guillermo, muchas gracias por todo lo que has aportado. ... 
"Pipo", muchas gracias por tu amabilidad y por tu dedicación, y sobre todo la elegancia a la hora de hacer los comentarios. 

Gracias a los miembros del tribunal (José Antonio Hódar, Juan Lorite, Lorena Gómez Aparicio, Paloma Ruiz Benito, Luis Matías, Alex Leverkus y Carolina Puerta-Piñero) por aceptar la invitación y dedicar tiempo a la lectura de esta tesis doctoral. Vuestras opiniones enriquecen enormemente el contenido de esta memoria doctoral.  

Gracias a  Pepa Rodríguez, Roberto Salomón, Brígida Simón que me proporcionaron diferentes fotografías. Gracias a CREADOS S.L. que me brindaron amablemente algunas ilustraciones para incluir en esta memoria. En especial agradezco a María Ángeles Lizana y a Rocío Guerrero. 

Bicha






Gracias a los compañeros con los que he corrido durante este tiempo. Ha sido una válvula de escape que me ayudó mucho. Gracias cuñao, por todos los raticos de trail que hemos pasado

Gracias a la familia de Hogares Nuevos, en especial a Babette y Arturo, Isa y Javi, Rocío y Julio, Julia y Pepe Mateos, Maria José y Gustavo, Inmaculada y Jose Barea, Inma Zamora y Antonio, María José y José Manuel, Inma y Antonio, Isa y Ramón. Gracias Jairo, Miguel Ángel, Jose Antonio. Gracias por compartir la vida durante este tiempo. Es un regalazo de Dios. Vuestras palabras, miradas, silencios, gestos, vuestro regalar vida, me han sostenido, especialmente en momentos muy difíciles. No me puedo olvidar de los prometedores vástagos Clara, Isa, Javi, Blanca, Julio, Jaime, Rocío, Pepe, Luis, Edu, Elías, Isaías, Evelio, XX, Miguel, Clara, Ana, Pablo, Isa, Álvaro, Gonzalo. Sois sin duda la sal y la luz del mundo, y es admirable las ganas tremendas que tenéis de mejorar este mundo. Vais a llegar muy lejos. Gracias por escucharme en los paseos por el campo y por soportarme en los campamentos en el Hotel del Duque, donde os calentaba la cabeza contándoos cosas de los robles.  



Gracias a Miguel Jiménez, y sobre todo a Octavio Díaz de la Guardia, que me han ayudado en los momentos más críticos. Octavio, 

... 
