% !TEX root = ../my-thesis.tex
%
\section*{Resúmen}\label{sec:resumen}
El estudio de la dinámica ecológica de las poblaciones localizadas en los límites de su distribución se considera esencial para establecer unas directrices de gestión adecuadas bajo las incertidumbres climáticas actuales. Las poblaciones de los márgenes posteriores suelen estar adaptadas a las condiciones ambientales locales en el límite de la amplitud ecológica de la especie, y a menudo muestran una persistencia a largo plazo. Las respuestas locales a los cambios ambientales pueden diferir de la respuesta media de la especie y tales diferencias pueden promover o dificultar la supervivencia de las poblaciones del margen posterior de distribución en las condiciones actuales de cambio global. Los robledales de \Qp en Sierra Nevada representan uno de los márgenes más meridionales del  área de distribución de esta especie. Están localizados en una región de montaña que ha servido de refugio para la especie. Al igual que otras formaciones forestales Mediterráneas, han estado sometidos a intensas presiones antropogénicas a lo largo del tiempo, lo que ha provocado una reducción de su extensión y una modificación de su composición florística y de sus patrones estructurales. Históricamente, los robledales han sido explotados en monte bajo para la obtención de leña, carbón vegetal, taninos. También se aclararon para generar grandes zonas de pastoreo. Todas estas actuaciones condujeron a una sobreexplotación de los robledales, de tal forma que la configuración actual de estas formaciones en Sierra Nevada parece depender en gran medida del uso del pasado. Sin embargo, a partir de la segunda mitad del siglo XX se produjo un abandono de las actividades tradicionales, resultando en una disminución de la presión antrópica sobre los ecosistemas forestales. Paradójicamente, muchos robledales presentan un estado de degradación avanzado (\emph{e.g.} problemas de regeneración, altas densidades, estancamiento en el crecimiento, entre otros). Estos problemas pueden verse agravados en el contexto actual de cambio climático, sobre todo teniendo en cuenta la alta vulnerabilidad de esta especie al cambio climático, y especialmente para las zonas situadas en el borde posterior de su área de distribución como es Sierra Nevada. Por tanto, entender la ecología de las poblaciones situadas en el borde posterior de su distribución es clave para evaluar la respuesta de la especie a las condiciones ambientales cambiantes. El objetivo general de esta tesis doctoral es analizar la dinámica de funcionamiento frente al cambio global de los robledales de \Qp situados en Sierra Nevada, una región montañosa que representa uno de los límites geográficos de su distribución, donde estas formaciones han sido objeto de intensas presiones antrópicas. 

En el \textcolor{ctcolormain}{capítulo \ref{sec:multivar}}, utilizando información ambiental de alta resolución e inventarios forestales, hemos determinado que la precipitación y la radiación son las variables más importantes que explican la distribución de las poblaciones de robledal dentro de su margen de distribución. Hemos identificado la existencia de tres grupos de poblaciones de robledal dentro de Sierra Nevada, debido principalmente a las condiciones microambientales. Encontramos una notable coincidencia entre la agrupación de las poblaciones derivada del análisis de las variables ambientales y la ordenación de las poblaciones según la composición de especies, lo que sugiere una relación entre la heterogeneidad de los factores ambientales y la variabilidad de la composición de especies para estos bosques. Esta diversidad de condiciones ecológicas para las poblaciones de \Qp situadas en este borde posterior, está en consonancia con los altos niveles de diversidad genética mostrados por las poblaciones de esta especie en Sierra Nevada. 

En el \textcolor{ctcolormain}{capítulo \ref{sec:coloniza}} hemos estudiado el patrón de colonización de los cultivos de montaña por parte de \Qp en dos localidades de Sierra Nevada. Para ello analizamos la estructura de la fuente de semillas (bosques circundantes), la abundancia de arrendajo \emph{Garrulus glandarius} (principal dispersor de bellotas de roble), y la abundancia de juveniles de \Qp en los cultivos abandonados. Asimismo caracterizamos en la medida de lo posible la historia de gestión antes y después del abandono de cultivos. Los resultados indican que se está produciendo una recolonización natural de las tierras de cultivo abandonadas por parte de \Qp en el borde posterior de su distribución. Encontramos diferencias en el patrón de colonización entre los sitios de estudio, que parecen estar relacionadas con las diferencias en la gestión anterior y posterior al abandono. Nuestros resultados demuestran que incluso en las actuales condiciones climáticas cada vez más secas, los bosques de \Qp son capaces de recuperar los antiguos campos de cultivo abandonados en el mismo nivel altitudinal donde el robledal es la vegetación potencial. Este proceso natural puede aportar soluciones para la conservación de la biodiversidad, y mejora la mitigación del cambio climático y la adaptación al mismo. 

En el \textcolor{ctcolormain}{capítulo \ref{sec:carbon}} realizamos una estimación de la capacidad de secuestro de carbono por parte de los robledales de \Qp en Sierra Nevada. Asimismo, estudiamos las tendencias temporales de la provisión de este servicio ecosistémico. Los robledales de Sierra Nevada, al igual que los del resto de la Península Ibérica, han experimentado un aumento de la biomasa total en las últimas décadas, y por tanto de su capacidad como sumidero de carbono. Los datos estimados de secuestro de Carbono para los robledales de Sierra Nevada son superiores a los observados para otras zonas, indicando el alto potencial como sumidero de carbono de estos robledales, a pesar de estar situados en el límite sur de su distribución. Estos resultados parecen estar relacionados con el hecho de que los robledales de Sierra Nevada son relativamente más jóvenes en comparación con otros robledales (debido al intenso uso antrópico al que han estado sometidos), y las condiciones de refugio en las que se encuentran los robledales de Sierra Nevada. Por otro lado, se encontraron diferencias  respecto al potencial de secuestro de Carbono entre las poblaciones de robledal de Sierra Nevada. Aquellas poblaciones que han estado sometidas a menos perturbaciones antropogénicas presentan una una mayor riqueza estructural, que se ve reflejado en valores más altos de biomasa, y por tanto, en un mayor potencial de secuestro de carbono. 

En el \textcolor{ctcolormain}{capítulo \ref{sec:dendro}}, combinando el estudio del verdor de la vegetación derivado de imágenes de satélite, con análisis dendrocronológicos de árboles adultos, hemos analizado la resiliencia a la sequía del crecimiento primario y secundario de los robledales de Sierra Nevada. Las tendencias de crecimiento reflejaron una fuerte influencia del uso del suelo en la estructura forestal actual. El crecimiento primario y secundario de esta especie, aún siendo muy sensible a la disponibilidad de agua, mostró una alta resiliencia a corto y largo plazo a los eventos de sequía. Los altos valores de resiliencia observados sugieren que las poblaciones de robledal de Sierra Nevada se encuentran en el borde posterior geográfico pero no climático ni ecológico. La resiliencia a la sequía que muestran los robledales de Sierra Nevada no es espacialmente homogénea, debido a las diferencias en las condiciones ecológicas y a los legados del uso del suelo. La gran variabilidad en las respuestas entre sitios muy cercanos geográficamente, parece indicar que las respuestas a la sequía son dependientes del sitio y pueden variar drásticamente en gradientes espaciales extremadamente estrechos, como los que ocurren en regiones de montaña. En los últimos años se ha observado una tendencia positiva en la productividad primaria y en el crecimiento secundario de los robledales de Sierra Nevada. Esta respuesta es diferente a la tendencia negativa de crecimiento que cabría esperar para las poblaciones situadas en el borde sur de su distribución, tal y como se ha observado para otras especies templadas y mediterráneas. 

En el \textcolor{ctcolormain}{capítulo \ref{sec:onto}} evaluamos cómo modificaciones en los patrones de disponibilidad de agua debido al cambio climático pueden afectar a la productividad de robledales de Sierra Nevada. Utilizando información derivada de imágenes de satélite junto con un sistema de ontologías, hemos observado la concurrencia de cambios en los patrones de innivación (disponibilidad de agua) y en la productividad primaria de los robledales de \Qp. En las zonas donde se ha detectado un adelanto en la fecha de fusión de la nieve, es decir, la nieve presenta una tendencia significativa a retirarse antes, también se ha observado un aumento significativo en la productividad primaria de verano para los robledales. Este acoplamiento entre las tendencias de producción primaria y las de duración de la nieve es más patente para las poblaciones de robledales occidentales de Sierra Nevada. Esta modificación en los patrones de disponibilidad de agua debido al cambio climático parece estar afectando a la productividad estacional de los robledales. 

Finalmente, en el \textcolor{ctcolormain}{capítulo \ref{sec:es}} realizamos una revisión de los servicios ecosistémicos proporcionados por los robledales de \Qp, combinando una revisión bibliográfica para todo el área de su distribución con un análisis espacio-temporal de la provisión de servicios ecosistémicos, usando los robledales de Sierra Nevada como caso de estudio. Los robledales proporcionan una gran cantidad de servicios ecosistémicos. Además del papel que presentan estos bosques como proveedores de servicios de regulación (\emph{e.g} sumidero de Carbono) o de provisión (\emph{e.g.} uso de su madera para el envejecimiento del vino), se ha puesto de manifiesto la existencia de un gran número de servicios ecosistémicos culturales proporcionados por los bosques de \Qp. El análisis espacio-temporal de la provisión de servicios ecosistémicos reveló diferencias en la oferta de servicios ecosistémicos entre las poblaciones de robledal de Sierra Nevada, siendo las poblaciones del sur las que presentan mayores valores de servicios de regulación y las del norte las que presentan mayores valores de servicios culturales. Observamos una variación temporal en el suministro de servicios ecosistémicos. Hasta mediados del siglo pasado, los servicios de provisión predominaban sobre los servicios de regulación y culturales. El abandono de las actividades tradicionales provocó una disminución de los servicios de provisión a favor de los servicios de regulación y, en las últimas décadas, de los servicios culturales. Nuestra recopilación de datos a escala local nos ha permitido cuantificar muchos de los servicios ecosistémicos prestados por los bosques de \Qp, lo cual proporciona a los gestores del territorio una información muy valiosa que puede ayudar en la planificación de actuaciones de gestión y conservación de esta formación forestal. 

\vspace*{20mm}

\section*{Abstract}\label{sec:abstract}
