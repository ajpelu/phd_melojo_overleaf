% !TEX root = ../my-thesis.tex
%
\chapter{Introducción}
\label{sec:intro}

\subsection{Objetivos}

\paragraph{Caracterización ambiental de los robledales de Sierra Nevada} \mbox{} \\
Realizaremos una caracterización ambiental del sistema de estudio (robledales) en Sierra Nevada. Queremos conocer el comportamiento de las diferentes poblaciones de robledal respecto a variables ambientales (climáticas, topográficas, etc.), composición, atributos forestales (estructurales) y funcionales. Concretamente, se pretende:

\begin{itemize}
	\item Caracterizar los robledales nevadenses respecto a variables abióticas y de estructura del bosque.
	\item Establecer los valores que definen el hábitat óptimo y marginal de las poblaciones de roble en Sierra Nevada.
	\item Identificar las variables abióticas que explican la distribución de los robledales en Sierra Nevada.
	\item Una vez identificadas las variables abióticas que condicionan la distribución de los robledales, nos interesa conocer si existen diferentes grupos determinados por variables abióticas. Intentaremos responder a la pregunta ¿que variables abióticas, o combinación de las mismas, discriminan mejor entre las poblaciones de robledal en Sierra Nevada?.
	\item Analizar si la discriminación en diferentes grupos, basada en características ambientales, se refleja en una discriminación basada en la composición de especies de las diferentes poblaciones así como de la estructura (atributos forestales) y el funcionamiento (regeneración y productividad). Es decir, ¿hay relación entre el agrupamiento basado en factores abióticos y el agrupamiento de composición, estructural y funcional?. 
\end{itemize}


\paragraph{Análisis del patron de abundancia de las masas de robledal en los últimos 50 años} \mbox{} \\
Analizaremos la evolución de las masas de robledal en Sierra Nevada en los últimos 50 años. Primero intentaremos responder a la pregunta: ¿ha ocurrido un proceso de densificación de las masas de robledal en Sierra Nevada en los últimos 50 años?. Partimos de la hipótesis de que los robledales, al igual que otras formaciones forestales mediterráneas , han experimentado un proceso de densificación en los últimos 50 años. Una vez respondida esta pregunta, queremos identificar cuales son los drivers que explican dicho proceso (factores climáticos, ambientales, antrópicos) y la importancia de cada uno de ellos en la densificación observada en cada momento temporal analizado. Posteriormente generaremos un modelo de abundancia para los robledales en Sierra Nevada.

\paragraph{Análisis del proceso de colonización de hábitats degradados próximos a las masas de robledal en Sierra Nevada}\mbox{} \\
Queremos analizar el proceso de colonización de hábitats degradados (campos de cultivo abandonados) próximos a las manchas de robledal y ver si existen si existen diferencias entre poblaciones de robledal en el límite sur de su distribución. Para ello nos enfocaremos en diferentes aspectos del proceso de colonización. En primer lugar analizaremos la estructura de la fuente semillera (bosque maduro) para explorar diferencias entre las poblaciones situadas en la zona norte y las situadas en la zona sur. También evaluaremos la evolución de las poblaciones del principal vector dispersante del roble, el arrendajo (Garrulus glandarius). Finalmente analizaremos el proceso de colonización en sí, es decir, los patrones de regeneración de robledal en diferentes manchas de cultivo abandonados, localizadas en zonas cercanas a las poblaciones de robledal.
Al final conoceremos mas en profundidad como es el patrón de colonización de hábitats marginales por parte del robledal y si existen diferencias entre las diferentes poblaciones. Comprender la expansión de los robledales hacia áreas márginales (cultivos abandonados) es crítico para poder desarrollar estrategias de gestión efectivas de los robledales.

\paragraph{Efectos del cambio climático en la productividad de los robledales en Sierra Nevada}\mbox{} \\
Se trata de evaluar como alteraciones en los patrones de disponibilidad hídrica debido al cambio climático pueden afectar a la productividad de esta formación forestal. Conocemos que los robledales presenta una estación de crecimiento bien definida y centrada en verano . Por ello es de interés evaluar la productividad de los robledales (utilizando índices de vegetación obtenidos a partir de imágenes de satélite) frente a cambios en los patrones de disponibilidad hídrica. Nos centraremos en analizar modificaciones en la cantidad de agua (disminución de la disponibilidad de agua debido a eventos de sequía) así como alteraciones en la distribución temporal de la disponibilidad de agua (\emph{p.ej.}: adelantos en la fusión de la cubierta de nieve).