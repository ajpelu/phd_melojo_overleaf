% !TEX root = thesis_qp_ajpelu.tex


% **************************************************
% Files' Character Encoding
% **************************************************
%\PassOptionsToPackage{utf8}{inputenc}
\usepackage[utf8]{inputenc}

\usepackage[english, spanish, es-tabla]{babel}

% **************************************************
% Information and Commands for Reuse
% **************************************************
\newcommand{\Qpw}{\textit{Quercus pyrenaica} Willd.\xspace}
\newcommand{\Qpy}{\textit{Quercus pyrenaica}\xspace}
\newcommand{\Qp}{\textit{Q. pyrenaica}\xspace}
\newcommand{\elev}{\textit{m.s.n.m.}\xspace}
\newcommand{\eleven}{\textit{m.a.s.l.}\xspace}
\newcommand{\mgha}{Mg ha$^{-1}$\xspace}
\newcommand{\ws}{W\textsubscript{stem}\xspace}
\newcommand{\wt}{W\textsubscript{total}\xspace}
\newcommand{\wro}{W\textsubscript{root}\xspace}
\newcommand{\wb}{W\textsubscript{b2}\xspace}
\newcommand{\wbs}{W\textsubscript{b2-7}\xspace}
\newcommand{\cs}{C\textsubscript{stem}\xspace}
\newcommand{\ct}{C\textsubscript{total}\xspace}
\newcommand{\cro}{C\textsubscript{root}\xspace}
\newcommand{\cb}{C\textsubscript{b2}\xspace}
\newcommand{\cbs}{C\textsubscript{b2-7}\xspace}





\newcommand{\thesisTitle}{Anáisis de la dinámica de los robledales (\Qpw) frente al cambio global en el límite sur de su distribución (Sierra Nevada)}
\newcommand{\thesisName}{Antonio J. P\'erez Luque}
\newcommand{\thesisSubject}{Biología Fundamental y de Sistemas}
\newcommand{\thesisDate}{Marzo, 2021}
\newcommand{\thesisVersion}{v1.1}

\newcommand{\thesisFirstReviewer}{Jane Doe}
\newcommand{\thesisFirstReviewerUniversity}{\protect{Clean Thesis Style University}}
\newcommand{\thesisFirstReviewerDepartment}{Department of Clean Thesis Style}

\newcommand{\thesisSecondReviewer}{John Doe}
\newcommand{\thesisSecondReviewerUniversity}{\protect{Clean Thesis Style University}}
\newcommand{\thesisSecondReviewerDepartment}{Department of Clean Thesis Style}

\newcommand{\thesisFirstSupervisor}{Regino Jesús Zamora Rodr\'iguez\xspace}
\newcommand{\thesisSecondSupervisor}{Francisco Javier Bonet Garc\'ia\xspace}

\newcommand{\thesisUniversity}{\protect{Clean Thesis Style University}}
\newcommand{\thesisUniversityDepartment}{Department of Clean Thesis Style}
\newcommand{\thesisUniversityInstitute}{Institute for Clean Thesis Dev}
\newcommand{\thesisUniversityGroup}{Clean Thesis Group (CTG)}
\newcommand{\thesisUniversityCity}{City}
\newcommand{\thesisUniversityStreetAddress}{Street address}
\newcommand{\thesisUniversityPostalCode}{Postal Code}

% **************************************************
% Debug LaTeX Information
% **************************************************
%\listfiles



% **************************************************
% Load and Configure Packages
% **************************************************
%\usepackage[english, spanish]{babel} % babel system, adjust the language of the content
\PassOptionsToPackage{% setup clean thesis style
    figuresep=space,%
    hangfigurecaption=false,%
    hangsection=true,%
    hangsubsection=true,%
    sansserif=false,%
    configurelistings=true,%
    colorize=full,%
    colortheme=bluegreen,%
    configurebiblatex=true,%
    bibsys=bibtex,%
    bibfile=referencias,%
    bibstyle=authoryear,%
    bibsorting=nyt, %
}{cleanthesis}
\usepackage{cleanthesis}

\hypersetup{% setup the hyperref-package options
    pdftitle={\thesisTitle},    %   - title (PDF meta)
    pdfsubject={\thesisSubject},%   - subject (PDF meta)
    pdfauthor={\thesisName},    %   - author (PDF meta)
    plainpages=false,           %   -
    colorlinks=false,           %   - colorize links?
    pdfborder={0 0 0},          %   -
    breaklinks=true,            %   - allow line break inside links
    bookmarksnumbered=true,     %
    bookmarksopen=true          %
}

% **************************************************
% Other Packages
% **************************************************
\usepackage{scrhack}
\usepackage{booktabs}			% fot tables 
\usepackage{ragged2e}			% justificar texto
\usepackage{rotating}			  % Rotar tablas
\usepackage{adjustbox}		% Para las tablas largas y rotadas
\usepackage{threeparttable}   % Idem 
\usepackage{longtable}          % tablas largas
\usepackage{float}          % para decidir donde colocar las tablas 
\usepackage{array}          % para las tablas 
\usepackage{multirow}
\usepackage{siunitx} % for scientific notation 
\usepackage{array} 
% **************************************************
% Commands for biblbliography
% **************************************************
\DefineBibliographyStrings{spanish}{%
	andothers = {\emph{et}\addabbrvspace \emph{al\adddot}}
}
